\section{Resource identification}
\label{sec4.1}

In this section the resources available/needed to perform the project will be exposed. These resources will be classified into three different categories: 
\begin{itemize}
\item Employees: People needed to achieve the objectives of the project. The employees will be provided by the members of the consortium. 
\item Materials: Hardware and software elements that will be used to achieve the project objectives.
\item Facilities: Special places and services (such as the testing room). 
\end{itemize}
A brief explanation of the resources needed will be done and a collection of all of them, including a Resource ID, will be shown in \ref{table_resourcesidentification}.\\
Regarding human resources, i.e. employees, these can be classified into three sub-groups as not all of them are in the same point on the learning curve.
\begin{itemize}
\item Senior: High on the learning curve. They are able to provide guidance on technical and management issues and offer a critical point of view of the actions of the project.
\item Average: They are able to perform activities on their knowledge field and arrive to conclusions without supervision.
\item Junior: Little experience in the field, the work done needs to be supervised by an average employee.
\end{itemize}
The employees will be chosen taking into account the roles and responsibilities and technical knowledge needed to perform the project.\\
Materials/hardware are also important in this project as a sensor and its modular system wants to be build and tested. Although not all the hardware can be clearly specified in this early stage of the project, blocks can be defined. Hardware is also needed to support the database and interaction platform in order to perform the objective of the project: urban development. The blocks are: 
\begin{itemize}
\item Payload building blocks: Hardware needed to build the sensor itself. It will depend on the type of sensor that needs to be build.
\item Hardware support system: Physical connection between the parts of the sensor and its modular system. It can consist of a multi-layer PCB.
\item Controllers: Chip, expansion card or stand-alone device to interface with the sensor and the other parts of the modular system. It can be a micro controller or an hybrid technology such as the combination of logic blocks with FPGA (Field-programmable Gate Array.
\item Memory modules: Additional SRAM/ROM memory blocks to complete the performance of the system.
\item Hosting package: Servers needed to host the end user and stakeholder platform for urban development.
\item Backup system: Archive of the interaction platform computer data. 
\end{itemize}
Regarding the facilities that will be used during the project, these are:
\begin{itemize}
\item Office: Needed to perform desktop tasks. All stakeholders can use their own offices.
\item Meeting rooms: To host meetings between the members of the consortium, employees of the same partner and between members of the consortium and clients. They will be provided by the consortium members. 
\item Research laboratory: Laboratories needed to do the research of the payload, modular system and interaction platform. They will be provided by the consortium members related with these tasks. 
\item Development centre: Centre where the development of payload, modular system and interaction platform will be carried out. They will be provided by the consortium members related with these tasks.
\item Testing room: Laboratories where the testing of the system in relevant environment will be done. They will be provided by the consortium members related with the testing.
\item Quality laboratory: Laboratories where the quality evaluation will be carried out. They will be provided by the consortium members related with quality assessment. 
\end{itemize}

\begin{longtable}{lll}
	
\toprule[2pt]

\textbf{Resource ID} & \textbf{Resource Description} & \textbf{Type of resource}  
	\\ \midrule[1.5pt] 
	\endhead
	
PM.M&Project Manager&Employee-Senior\\
PM.S&Project Manager Secretary&Employee-Average\\
FM.M&Financial Manager&Employee-Senior\\
FM.A&Financial Manager Assessor&Employee-Average\\
SPM.M&Stakeholders and Procurement Manager&Employee-Senior\\
SPM.A&Stakeholders and Procurement Manager Assessor&Employee-Average\\
ScTM.M&Scope and Time Manager&Employee-Senior\\
ScTM.A&Scope and Time Manager Assessor&Employee-Average\\
RM.M&Risk Manager&Employee-Senior\\
RM.A&Risk Manager Assessor&Employee-Average\\
QM.M&Quality Manager&Employee-Senior\\
QM.A&Quality Manager Assessor&Employee-Senior\\
MCM.M&Marketing and Communications Manager&Employee-Senior\\
MCM.A&Marketing and Communications Manager Assessor&Employee-Average\\
TM&Tecnhical Manager&Employee-Average\\
RD.A&Research and development assessor&Employee-Average\\
LB.A&Legal and Business Assessor&Employee-Average\\
SD.S&System development engineer&Employee-Senior\\
SD.A&System development engineer&Employee-Average\\
SD.J&System development engineer&Employee-Junior\\
ST.S&System testing engineer&Employee-Senior\\
ST.A&System testing engineer&Employee-Average\\
ST.J&System testing engineer&Employee-Junior\\
AD.S&Application development manager&Employee-Senior\\
AD.A&Application development technician&Employee-Average\\
AD.J&Application development technician&Employee-Junior\\
SOFT.1&Microsoft Office&Material\\
SOFT.2&LaTex&Material\\
SOFT.3&GitHub&Material\\
SOFT.4&Trello&Material\\
SOFT.5&Solidworks&Material\\
SOFT.6&PostgreSQL&Material\\
SOFT.7&Live Plan&Material\\
SOFT.8&Wix&Material\\
SOFT.9&Jitsi&Material\\
SOFT.10&Final Cut Pro&Material\\
HARDW.1&Payload building blocks&Material\\
HARDW.2&Hardware support system&Material\\
HARDW.3&Sensor interface&Material\\
HARDW.4&Controllers&Material\\
HARDW.5&Memory modules&Material\\
HARDW.6&Hosting package&Material\\
HARDW.7&Backup system&Material\\
OFF& Office & Facilities\\
MR& Meeting room& Facilities\\
CH& Conference Hall& Facilities\\
RL & Research laboratory & Facilities\\
DC & Development centre & Facilities\\
TR& Testing room & Facilities\\
QL& Quality laboratory& Facilities\\


\bottomrule[2pt]

\caption{Resources identification}
\label{table_resourcesidentification}	
\end{longtable}

