\chapter{Estimate activity duration}

In this section an estimate activity duration is performed. In order to perform the most accurate estimation possible three different methods, which are explained below, have been applied depending on the character of each task. 

A brief explanation of the three commented methods including the type of tasks that have been estimated with each one is carried out:

\textbf{Parametric Estimate} \\
This estimation technique, that has been used to estimate the duration of commercial and administration tasks, uses an algorithm based on historical data and project parameters. The algorithm used consists in: 

\begin{equation}
\centering
\notag Duration Estimate=\frac{Effort Days}{Resource Quantity * Available Factor * Performance Factor}
\end{equation}

The parameters used have been established as follows:

\begin{itemize}
	
	\item \textbf{Effort Days:} States the necessary days to complete the task.
	
	\item \textbf{Resource Quantity:} Determines the number of resources (people) assigned at the respective task. This parameter has been already established on section \ref{sec4.2}.
	
	\item \textbf{Available Factor:} Determines the availability of the resources. This parameter has been established taking into account the overlap of tasks assigned to each resource.

	\item \textbf{Performance Factor:} Determines the ability of the resource assigned to perform the task. This parameter has been established taking into account if the resource was a Senior-Employee, an Average-Employee, a Junior-Employee or a combination of them (in section \ref{sec4.1} each type of employee has been defined).
	
\end{itemize} 

\textbf{Analogous Estimate}

The analogous estimation technique is based on the knowledge about the activity duration of previous similar projects. Hence, the duration of the current project completely relies on the duration of the previous one and the weight this current activity has compared to the previous one. 
\begin{equation}
\centering
\notag Duration Estimate=Previous activity Duration * Multiplier
\end{equation}

Analogous estimating can be less accurate than other estimation methods when the previous activities are not similar enough. Therefore, it has been used to estimate the duration of the management activities because it is possible to obtain a reliable estimation from the management timings of previous projects. \\

\textbf{Three Point Estimate}

The third method used to estimate the duration of the activities is the three-point estimation. This method takes into account the uncertainties and risks in order to provide an expected duration of each activity. 

The expected duration is calculated by using a Beta Distribution which gives more weight to the most likely duration of the activity than the other parameters have because it is the most realistic one.

\begin{equation}
\centering
\notag Beta Distribution=\frac{Optimistic + (Most Likely) * 6 + Pessimistic}{6}
\end{equation}

In fact, the parameters from which each activity duration is calculated are defined as: 

\begin{itemize}

\item \textbf{Most Likely:} this estimate is based on the duration of the activity in a realistic way, by taking into account resources available and productivity for the corresponding activity. 

\item \textbf{Optimistic:} this estimate is based on the best-scenario for the activity. 

\item \textbf{Pessimistic:} this estimate is based on the worst-case scenario for the activity. 

\end{itemize}

This method has been used to estimate the duration of the activities that are technical, because there is no data about previous projects due to the fact it is a state of the art project. However, it is a good method because by knowing the resources available and the productivity of the team, realistic expectations of the availability for the activity and its workload are done. \\

\textbf{Estimate activity duration}
\begin{longtable}[H]{C{1.5cm} C{2cm} C{2cm} C{2cm} C{2.4cm} C{2.4cm} }

	\toprule[2pt]
	 \multicolumn{6}{c}{\textbf{Parametric Estimates}}\\ \bottomrule[2pt]
	\toprule[2pt]

	\textbf{WBS-ID} &  \textbf{Effort Days}  & \textbf{Resource Quantity} & \textbf{\% Available} &\textbf{Performance Factor} & \textbf{Duration Estimate (days)}\\ 
	
	\midrule [1.5pt]
	\endhead

		2.1.1 & 54 & 2 & 100 & 0.9 & 30\\ \midrule
		2.1.2 & 1500 & 2 & 100 & 0.9 & 830\\ \midrule
		2.2.1 & 4 & 2 & 60 & 0.9 & 4 \\ \midrule
		2.2.2 & 3 & 2 & 60 & 0.9 & 3 \\ \midrule
		2.2.3 & 7 & 2 & 80 & 0.9 & 5 \\ \midrule
		2.2.4 & 6 & 2 & 70 & 0.9 & 5 \\ \midrule
		2.2.5 & 5 & 2 & 100 & 0.9 & 3 \\ \midrule
		2.3 & 1500 & 2 & 90 & 0.95 & 880 \\ \midrule
		6.1.1 & 12 & 3 & 60 & 0.85 & 8 \\ \midrule
		6.1.2 & 11 & 4 & 60 & 0.85 & 5 \\ \midrule
		6.1.3 & 12 & 3 & 50 & 0.85 & 10 \\ \midrule
		6.2 & 27 & 5 & 80 & 0.85 & 8 \\ \midrule
		7.1 & 51 & 3 & 100 & 0.85 & 20 \\ \midrule
		7.2.1 & 180 & 3 & 80 & 0.85 & 90 \\ \midrule
		7.2.2 & 1475 & 3 & 75 & 0.85 & 770 \\ \midrule
		7.3 & 2200 & 6 & 50 & 0.85 & 860 \\ \midrule
		7.4.1 & 1650 & 3 & 75 & 0.85 & 860 \\ \midrule
		7.4.2 & 1650 & 3 & 75 & 0.85 & 860 \\ \midrule
	
	\\ \bottomrule[2pt]
	\caption{List of Parametric Estimates}

\end{longtable}

	
\begin{longtable}[H]{C{1.5cm} C{2.9cm} C{1.7cm} C{2.7cm} C{1.8cm} C{1.8cm} }

	\toprule[2pt]
	\multicolumn{6}{c}{\textbf{Analogous Estimates}}\\ \bottomrule[2pt]
	\toprule[2pt]
	
	\textbf{WBS-ID} &  \textbf{Previous Activity}  & \textbf{Previous Duration} & \textbf{Current Activity} &\textbf{Multiplier} & \textbf{Duration Estimate}\\ 
	
	\midrule [1.5pt]
	\endhead
	
		1.1 & Previous project Management Plan & 23 & Project management plan & 0.9 & 20\\ \midrule
		1.2 & Previous project Monitoring & 980 & Monitoring of the project & 0.9 & 880\\ \midrule
		1.3 & Previous project Annual reporting & 1250 & Annual Reporting & 0.7 & 880\\ \midrule
		1.4 & Previous project Risk Management implementation & 1100 & Project implementation of risk management & 0.8 & 880\\ \midrule
	
	\\ \bottomrule[2pt]
	\caption{List of Analogous Estimates}

\end{longtable}

\begin{longtable}[H]{C{1.5cm} C{2cm} C{2cm} C{2cm} C{3cm} C{2cm} }

	\toprule[2pt]
	\multicolumn{6}{c}{\textbf{Three Point Estimates}}\\ \bottomrule[2pt]
	\toprule[2pt]

	\textbf{WBS-ID} &  \textbf{Optimistic Duration}  & \textbf{Most Likely Duration} & \textbf{Pessimistic Duration} &\textbf{Weighting Equation} & \textbf{Expected Duration Estimate}\\ 
	
	\midrule [1.5pt]
	\endhead

		3.1.1 & 20 & 30 & 40 & (o+4m+p)/6 & 30\\ \midrule

		3.1.2 & 15 & 23 & 40 & (o+4m+p)/6 & 25\\ \midrule

		3.2.1 & 20 & 24 & 35 & (o+4m+p)/6 & 25\\ \midrule
	
		3.2.2 & 22 & 28 & 45 & (o+4m+p)/6 & 30\\ \midrule	
	
		3.3.1 & 15 & 18 & 30 & (o+4m+p)/6 & 20\\ \midrule	
	
		3.3.2 & 32 & 40 & 50 & (o+4m+p)/6 & 40\\ \midrule	
	
		4.1.1.1 & 58 & 68 & 88 & (o+4m+p)/6 & 70\\ \midrule
		
		4.1.1.2 & 40 & 48 & 65 & (o+4m+p)/6 & 50\\ \midrule
		
		4.1.2 & 45 & 60 & 75 & (o+4m+p)/6 & 60\\ \midrule
		
		4.1.3.1 & 15 & 18 & 30 & (o+4m+p)/6 & 20\\ \midrule

		4.1.3.2 & 30 & 39 & 55 & (o+4m+p)/6 & 40\\ \midrule
		
		4.1.3.3 & 15 & 18 & 30 & (o+4m+p)/6 & 20\\ \midrule
		
		4.2.1 & 230 & 255 & 310 & (o+4m+p)/6 & 260\\ \midrule
		
		4.2.2 & 230 & 255 & 310 & (o+4m+p)/6 & 260\\ \midrule
		
		4.2.3 & 230 & 255 & 310 & (o+4m+p)/6 & 260\\ \midrule
		
		5.1.1 & 180 & 195 & 240 & (o+4m+p)/6 & 200\\ \midrule
		
		5.1.2 & 130 & 145 & 185 & (o+4m+p)/6 & 150\\ \midrule
		
		5.1.3 & 80 & 97 & 130 & (o+4m+p)/6 & 100\\ \midrule
		
		5.2 & 40 & 62 & 72 & (o+4m+p)/6 & 60\\ \midrule
		
		5.3 & 46 & 58 & 80 & (o+4m+p)/6 & 60\\ \midrule
		
		5.4 & 30 & 45 & 60 & (o+4m+p)/6 & 45\\ \midrule
		
		5.5 & 60 & 76 & 110 & (o+4m+p)/6 & 80\\ \midrule
		
		5.6 & 45 & 58 & 80 & (o+4m+p)/6 & 60\\ \midrule
		
    \bottomrule[2pt]
	\caption{List of Three Point Estimations}

\end{longtable}
	
	