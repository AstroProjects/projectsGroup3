\section{Level of accuracy}

According to the resource identification, the different positions from different resources have been rearranged in three groups in order to be more efficient when estimating the personnel costs. The three groups are: senior employees, average employees and junior employees. This classification is based in a Robert Walters consulting study \cite{} which shows that most of the manager positions are occupied by senior employees, while secretary positions and assessor positions are mostly occupied by average and junior employees respectively. By doing this three-group classification, the cost-estimating process has been highly simplified without compromising its precision. The cost associated to each of these groups has also been extracted from the aforementioned study, more specifically, an average salary from many different European countries has been selected since the project is to be developed at a European extent.

Software costs have been calculated according to the official price given by the developer (in price per user, price per month or price per license). In some cases, where different pricing plans can apply when acquiring the license, the middle cost option has been considered.   

When it comes to facilities cost estimation (offices, meeting rooms, research laboratories, etc.), a deep research in real estate and specialized pricing has provided an approximation to what the costs of renting and/or owning the different facilities will be.

The next section contains full detail of what the costs per activity are as well as the amount of resources needed by every single one of them. 
