\section{Quality Assurance Approach}
One of the most important parts of the project is to ensure high quality levels in all its sections in order to provide a final product that meets the expectations of possible future cutomers. In this section, the procedures and methods to ensure this high quality are detailed. 

At this point, it is important to recall the defined high-level technical requirements defined previously in the Project Charter:
\begin{table}[H]
	\centering
	\begin{tabular}{l p{13.3cm}}
		
		\toprule[2pt]
		
		\textbf{Item} &  \textbf{Description}\\
		
		\midrule [1.5pt]
		
		T1 & Ensure the endurance of the overall system.\vspace{0.2cm}\\
		
		\midrule
		
		T2 & Readiness for operational services.\vspace{0.2cm}\\
		
		\midrule
		
		T3 & Ability to detect greenhouse gases.\vspace{0.2cm}\\
		
		\midrule
		
		T4 & Ability to detect weather patterns for proper weather forecasting applications.\vspace{0.2cm}\\
		
		\midrule
		
		T5 & Ability to perform a high precision terrain mapping for urban applications.\vspace{0.2cm}\\
		
		\midrule
		
		T6 & The system must have a program for automatic updates and self-revision of possible issues.\vspace{0.2cm}\\
		
		\midrule
		
		T7 & Availability of real-time information with a maximum delay of 1 second.\vspace{0.2cm}\\
		
		\midrule
		
		T8 & 15\% increase of the reliability and precision of results compared to current technologies.\vspace{0.2cm}\\
		
		\bottomrule[2pt]
		
	\end{tabular}
	\caption{Technical requirements}
\end{table}

The quality assurance will be applied in different steps of the project. Before manufacturing the prototype, a quality procedure must be applied over the final design to ensure it meets the requirements of the project. The procedures executed in the manufacture of the prototype must be validated guarantee that they are suitable for the manufacture of the product. Finally, the final product must be revised to ensure it fulfills the expected specifications. This validations will contain methods to check the quality of the software and the hardware of the project.
