\section{Quality Assurance Approach}
One of the most important parts of the project is to ensure high quality levels in all its sections in order to provide a final product that meets the expectations of the possible future customers. In this section, the procedures and methods to ensure this high quality are detailed. 

At this point, it is important to recall high-level technical requirements defined previously in the Project Charter:
\begin{table}[H]
	\centering
	\begin{tabular}{c p{13.3cm}}
		
		\toprule[2pt]
		
		\textbf{Item} &  \textbf{Description}\\
		
		\midrule [1.5pt]
		
		T1 & Ensure the endurance of the overall system.\vspace{0.2cm}\\
		
		\midrule
		
		T2 & Readiness for operational services.\vspace{0.2cm}\\
		
		\midrule
		
		T3 & Ability to detect greenhouse gases.\vspace{0.2cm}\\
		
		\midrule
		
		T4 & Ability to detect weather patterns for proper weather forecasting applications.\vspace{0.2cm}\\
		
		\midrule
		
		T5 & Ability to perform a high precision terrain mapping for urban applications.\vspace{0.2cm}\\
		
		\midrule
		
		T6 & The system must have a program for automatic updates and self-revision of possible issues.\vspace{0.2cm}\\
		
		\midrule
		
		T7 & Availability of real-time information with a maximum delay of 1 second.\vspace{0.2cm}\\
		
		\midrule
		
		T8 & 15\% increase in the reliability and precision of the results compared to current technologies.\vspace{0.2cm}\\
		
		\bottomrule[2pt]
		
	\end{tabular}
	\caption{Technical requirements}
\end{table}

The quality assurance will be applied in the different steps of the project in order to obtain the desired results. These steps are:
\begin{itemize}
	\item  Before manufacturing the prototype. Quality procedures must be applied over the final design to ensure it meets the requirements of the project. These quality procedures include individual part structural  and assembly testing (ISO 10786:2011), electrical testing (ISO 17.220.20) and equipment laboratory testing according tho standarised norms.
	\item During the manufacture. The procedures applied in the manufacturing of the prototype must be validated to guarantee that they are suitable for the manufacturing of the product according to ISO 10794:2011.
	\item Final product validation. The final product must be revised to ensure it fulfils the expected specifications. These validations will contain methods to check the quality of the software and the hardware of the project according to ISO 9000:2005 validation requirements.
\end{itemize} 
Now that the quality needed has been specified, in the following sections the methods to control the quality and to improve the quality plan will be described.