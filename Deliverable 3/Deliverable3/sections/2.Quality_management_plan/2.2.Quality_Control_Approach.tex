\section{Quality Control Approach}

The quality control approach of the project is divided in three main areas:
\begin{itemize}
	\item Documentation quality plan
	\item Technical quality plan
	\item Software quality plan
\end{itemize}

\subsection{Documentation quality plan}
All the documentation of the project has to follow a strict quality plan in order to ensure that no information is lost. To do so, there is a series of steps that have to be followed:
\begin{enumerate}
	\item Definition of the document
	\begin{itemize}
		\item Define the type of document and its content as well as the standards that it has to follow.
		\item Define the responsible of the document, the team that is going to work in it and the team that is going to verify it.
		\item Define the deadline of the document as well as any milestone that may be related to it.
	\end{itemize}
	\item Redaction of the document: There may be some periodic quality controls while the document is in progress to ensure that the quality plan is met.
	\item Review and approval: Once the document is finished, the responsible of that deliverable should perform the following tasks regarding the document:
	\begin{itemize}
		\item Spell check.
		\item Consistency.
		\item References up to date.
		\item Check that the deliverable follows the acceptance criteria.
	\end{itemize}
	Then, the document can be delivered to the quality department. It will verify that the documentation follows the quality standards defined by the company. With the aim of guaranteeing a complete and trustful review, there should be at least two independent reviewers and they should not have been involved in the making of that document.
	If there is any review comment, it should be communicated to the deliverable responsible, since he/she has the final responsibility that the document meets the acceptance criteria.
\end{enumerate}
This documentation quality plan refers to the deliverables but also to the internal documents of the company.

\subsection{Technical quality plan}
Since part of the project consists in the design and construction of a prototype, it is necessary to ensure that it meets all the quality requirements to guarantee its proper operation. In order to do that, the following steps are defined:
\begin{enumerate}
	\item Definition of the quality plan: Before beginning with the design, a quality plan has to be done in order to define the acceptance criteria.
	\item Design: Once the plan is finished and the design phase starts, some quality controls have to be done periodically to guarantee that the design complies the requirements and follows the quality plan previously defined.
	\item Prototype and validation: During the construction of the prototype all the components and the production stages have to meet the acceptance criteria defined at the quality plan. Then, when the prototype is ready, a validation must be done in order to check that it fulfils all the requirements of the project as well as to verify that it complies the quality plan. This validation process has to follow the standards given by the industry.
\end{enumerate}

\subsection{Software quality plan}
The project not only consists of a prototype that should be constructed, but it also has a software that has to be verified. The following steps are defined to guarantee a satisfactory design of the implementation platform:
\begin{enumerate}
	\item Definition of the quality plan: Before starting with the coding, a software quality plan has to be defined. This document will set some standards that will have to be followed in the making of the interaction platform, such as coding and comment standards, to ensure a correct flow of information between the people who work on it as well as to avoid possible errors. It will also define the acceptance criteria that has to be met by the interaction platform.
	\item Coding phase: During the design phase, every modification of the code will have to be registered indicating the date and a description of the changes. Then, a review of the latest modifications should be done before making them definitive.	If an error is detected, it has to be immediately reported to the responsible of the software development. Then, an engineer will be assigned to solve it, and he/she will report it once the problem is solved.
	\item Implementation and validation: Once the interaction platform is operative, a validation has to be performed in order to ensure that it fulfils all the requirements of the project as well as to verify that it complies the software quality plan. This validation process has to follow the standards given by the industry.
\end{enumerate}