\section{Quality Control Approach}

%Describe the processes, procedures, methods, tools, and techniques that will be used in performing quality control activities. Define the quality standards that you will use for the previous requirements.

The quality control plan of the project is divided in three main areas:
\begin{itemize}
	\item Documentation quality plan
	\item Technical quality plan
	\item Software quality plan
\end{itemize}

\subsection{Documentation quality plan}
All the documentation of the project has to follow a strict quality plan in order to ensure that no information is lost. This plan refers to the deliverables but also to the internal documents of the company. The processes that have to be followed are:
\begin{enumerate}
	\item Definition of the document
	\begin{itemize}
		\item Define the type of document and its content as well as the standards that it has to follow.
		\item Define the responsible of the document, the team that is going to work in it and the team that is going to verify it.
		\item Define the deadline of the document as well as any milestone that may be related to it.
	\end{itemize}
	\item Redaction of the document: While the document is in progress there may be some periodic quality controls to ensure that the quality plan is met.
	\item Review and approval: Once the document is finished, it is delivered to the quality department. They have to verify that the documentation follows the quality standards defined by the company.
\end{enumerate}

%\begin{longtable}[H]{>{\raggedright\arraybackslash}p{2.7cm} p{5cm} p{2.7cm} p{2.7cm}}
%	
%	\toprule[2pt]
%	
%	\textbf{Project Deliverable} &  \textbf{Deliverable Quality Standards} & \textbf{Quality control activity} &  \textbf{Frequency}  \\
%	
%	\midrule [1.5pt]
%	\endhead
%	
%	Project Management Plan & miau & miau & miau \\
%	
%	\midrule
%	
%	Business Plan & miau & miau & miau \\
%	
%	\midrule
%	
%	Communication Plan & miau & miau & miau \\
%	
%	\midrule
%	
%	Payload State of the Art & miau & miau & miau \\
%	
%	\midrule
%	
%	Modular System State of the Art & miau & miau & miau \\
%	
%	\midrule
%	
%	Space Applications State of the Art & miau & miau & miau \\
%	
%	\midrule
%	
%	Payload Preliminary Design & miau & miau & miau \\
%	
%	\midrule
%	
%	Modular System Preliminary Design & miau & miau & miau \\
%	
%	\midrule
%	
%	Interaction Platform Preliminary Design & miau & miau & miau \\
%	
%	\midrule
%	
%	Payload Final Design & miau & miau & miau \\
%	
%	\midrule
%	
%	Modular System Final Design & miau & miau & miau \\
%	
%	\midrule
%	
%	Sensors Data Fusion Software Report & miau & miau & miau \\
%	
%	\midrule
%	
%	Interaction Platform Final Design & miau & miau & miau \\
%	
%	\midrule
%	
%	Data Processing Software Report & miau & miau & miau \\
%	
%	\midrule
%	
%	Validation & miau & miau & miau \\
%	
%	\midrule
%	
%	Final Report & miau & miau & miau \\
%	
%	\bottomrule[2pt]
%	
%	\caption{List of quality activities for the project deliverables}
%	
%\end{longtable}

\subsection{Technical quality plan}
Since the project consists in the design and construction of the prototype, it is necessary to ensure that the product of the project meets all the quality requirements. To do so, before beginning with the design, a quality plan has to be defined. Once the plan is finished and the design phase starts, there are some procedures that will have to be done regularly:
\begin{itemize}
	\item Check that the design fulfils the requirements of the project.
	\item Check for possible incompatibilities between the payload and the modular system.
	\item Review that the milestones are met in the given deadlines.
\end{itemize}
Finally, when the design is over and the prototype is constructed, a validation must be done in order to check that it fulfils all the requirements of the project as well as to verify that it complies the quality plan. This validation process has to follow the standards given by the industry.

\subsection{Software quality plan}
The project not only consists of a prototype that should be constructed, but it also has a software that has to be verified. In order to do so, before stating with the coding, a software quality plan has to be defined. According to this document, some standards have to be followed in the making of the interaction platform, such as coding and comment standards, to ensure a correct flow of information between the people who work on it as well as to avoid possible errors.
During the design phase, some procedures will be done regularly:
\begin{itemize}
	\item Check that the standards are being followed.
	\item Avoid possible incompatibilities between the interaction platform and the payload or modular system.
	\item Review the latest modifications before making them definitive.
\end{itemize}
Once an error is detected, it has to be immediately reported to the responsible of the software development. Then, an engineer will be assigned to solve it, and he/she will report it once the problem is solved.

Finally, once the interaction platform is operative, a validation has to be performed in order to ensure that it fulfils all the requirements of the project as well as to verify it complies the software quality plan. This validation process has to follow the standards given by the industry.