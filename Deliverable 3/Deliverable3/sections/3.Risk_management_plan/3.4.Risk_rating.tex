\section{Risk rating}
As already mentioned, risk rate is determined through probability and impact scores. In fact, it is the result of multiplying both scores. Hence, to identify a risk’s position in the matrix, first it is necessary to assess probability and impact score as explained in sections \ref{3.1} and \ref{3.2}.
The previously defined matrix, represents impact as an overall score but in our case, different impact scores have been defined depending on the project objective that is threatened (scope, schedule, or cost). Hence, to determine the general impact grade the following equation is defined:
\begin{equation}
I_{general}=\sum_{i}(W_i\cdot I_i)
\end{equation}
Where:
\begin{itemize}
	\item $i$ represents the different types of impact (scope, schedule, cost)
	\item $W_i$ represents the importance or weight (from 0 to 1) of each of the impact types and it is satisfied that $Wscope + Wschedule + Wcost = 1$
	\item $I_i$ represents the impact score of each of the types (from 0 to 5)
\end{itemize}

Consequently, the overall impact will have a value of (0-5] calculated doing a balance between each type of impact importance.
Regarding the weights defined for this project, it has been decided that cost is the most important, followed by scope and finally, the schedule. Hence, the values assigned are the ones shown  below:
	$$W_{scope}=0.3$$
	$$W_{schedule}=0.2$$
	$$W_{cost}=0.5$$
Once the general impact is calculated, the risk rating is defined as:
$\text{Risk Rating}=\text{Probability Score}\times \text{Impact Score}$

