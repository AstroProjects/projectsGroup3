\section{Communication process}

[PONER UNA INTRODUCCION]

\subsection{Informal}

Informal communications consist of e-mail, conversations, or phone calls and serve to supplement and enhance formal communications. Due to the varied types and ad-hoc nature of informal communications, they are not discussed in this plan.

\subsection{Formal}

The DEOS-UD Project will engage in various types of formal communication. The general types and their purpose are described below as “Status Meetings” and “Status Reports”.

\subsubsection{Status Meetings}

There are five basic types of status meetings for the DEOS-UD Project:
 \begin{itemize}
	\item Status meetings internal to the DEOS-UD business team to discuss assignments, activities, and to share information
	\item Status meetings and reports between the DEOS-UD business team, and the technical project team
	\item Advisory Committee meetings with the project stakeholders, and project manager to review progress, risks, and issues
	\item Status meetings and reports between the DEOS-UD project manager and the steering committee
	\item Status meetings and reports to stakeholders, such as oversight agencies
 \end{itemize}

\subsubsection{Status Reports}

A variety of status reports will be produced during the project. The status reports will be produced on regular intervals to provide stakeholders project information on the status and progress of the DEOS-UD project. At a minimum the reports will contain:
 \begin{itemize}
	\item Project status on major activities
	\item Project schedule
	\item Budget and cost tracking
	\item Status of issues and risks
	\item Health status	
	\item Status of action items, if applicable.
	\item Future or planned activities 
 \end{itemize}

The intent of the status reports is to inform stakeholders of the project’s progress and keep them actively involved in the project. The information provided will contain enough detail to allow stakeholders to make informed decisions and maintain oversight of the project.

\subsection{External Communication}

Although internal communication is very important for the proper development of the project, we must not forget that external communication is also crucial in a project of this magnitude. Having a good dissemination plan involves explaining how the outcomes of the project will be shared with stakeholders, relevant institutions, organisations, and individuals. \\
In order to achieve the proposed objectives in terms of external communication, the process of dissemination will be focused  in two different ways depending on whether you want to reach the general public or aerospace sector.

\subsubsection{General public}

It is important to find the proper way to reach the less specialized public in the aeroespace field. In order to achieve the maximum diffusion of the project in this sector,  the following resources will be used.

\begin{itemize}
\item{ 
	Social Networking. Social networks are the best way to reach the widest possible audience. Posting regularly is also crucial to keep people interested in the project. Some of the platforms that will be used during the project development are: Twitter, Facebook and Instagram. There will be at least one update a week in order to keep people informed of the progress of the project.
}
\item {
	Website. A project website is one of the most versatile dissemiation tools and will help to reach people that are not so familiar with social networks. It can contain information intended for different types of public. As in the previous case, it has to be kept updated.
}
\end{itemize}

\subsubsection{Aerospace sector}

\begin{itemize}
\item\textbf{PONER CUANTOS VAMOS HA HACER O ALGÚN EJEMPLO?}
\item{
	Trade shows. Trade shows, fairs and exhibitions are a great way to get in close contact with people from other regions and countries that you would ordinarily never be face to face with. They are also helpfull in terms of finding new prospects, nurture current client relationships and stay up to date on the latest industry developments. 
}
\item {
	Conferences. National and international conferences will help to share the achievements of the project with specialists of the field.
}
\item {
	Journal Articles
	
	
	To promote project ideas, concepts and results in scientific research and applied research communities, and get feedback from relevant stakeholders in these communities
	
	Any and every opportunity should be taken to get articles published about the project. Consider peer reviewed journals in relevant disciplines near the end of the project when you have data and results to report. Make sure to post a copy of all publications on your website. 
}
\end{itemize}