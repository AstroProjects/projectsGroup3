\chapter{Formato del texto}

Las opciones más importantespara dar formato al texto ya vienen, tanto en TeXmaker como en TeXstudio, en una columna a la izquierda del editor de texto (en la parte donde escribes, pues justo a la izquierda, es la columna que tiene una B, I, U arriba). Se puede hacer clic en la opción y luego introducir el texto, o seleccionar el texto y hacer clic.

\textbf{Texto en negrita}. Se crea con \verb!\textbf{Texto en negrita}!

\textit{Texto en cursiva}. Se crea con \verb!\textit{Texto en cursiva}!

\underline{Texto subrayado}. Se crea con \verb!\underline{Texto subrayado}!

Texto justificado para que ocupe todo el ancho de la hoja, siempre-y-cuando-no-sea-el-final-del-párrafo. Por defecto todos los textos se justifican automáticamente, de modo que no hay que preocuparse.

\begin{flushleft}
	Texto ajustado a la izquierda, de modo que aunque cuando se cambia de línea, no se ensancha la línea para que ocupe todo el ancho de la hoja, y da lugar a un borde derecho irregular. \newline
	Para esto se pone el comando \verb!\begin{flushleft}! \newline
	A continuación se escribe texto. \newline
	Y se cierra con \verb!\end{flushleft}!
\end{flushleft}

\begin{flushright}
	Texto ajustado a la derecha. \\
	Para esto se pone el comando \verb!\begin{flushright}! \\
	A continuación se escribe texto. \\
	Y se cierra con \verb!\end{flushright}! \\
	Para estos casos, si se quiere empezar una línea justo debajo, mejor con \verb!\\!
\end{flushright}

\begin{center}
	Texto centrado.\\
	Para esto se pone el comando \verb!\begin{center}! \\
	A continuación se escribe texto. \\
	Y se cierra con \verb!\end{center}!
\end{center}

Si se quiere generar un espacio vacío \hspace{1cm} tanto delante como en medio de un párrafo, se puede usar \verb!\hspace{1cm}!, que en este caso genera un espacio vacío de 1cm

A continuación vienen ejemplos de texto a distintos tamaños de letra. 

{\tiny Texto. Se hace con \verb!{\tiny Texto}!}

{\scriptsize Texto. Se hace con \verb!{\scriptsize Texto}!}

{\footnotesize Texto. Se hace con \verb!{\footnotesize Texto}!}

{\small Texto. Se hace con \verb!{\small Texto}!}

{\normalsize Texto. Se hace con \verb!{\normalsize Texto}!}

{\large Texto. Se hace con \verb!{\large Texto}!}

{\Large Texto. Se hace con \verb!{\Large Texto}!}

{\LARGE Texto. Se hace con \verb!{\LARGE Texto}!}

{\huge Texto. Se hace con \verb!{\huge Texto}!}

{\Huge Texto. Se hace con \verb!{\Huge Texto}!}

Finalmente, también se puede cambiar el color del texto, usando \verb!{\color[HTML]{XXXXXX} Texto}!, donde XXXXXX es el código del color

{\color[HTML]{FF0000} Texto. Se hace con \verb!{\color[HTML]{FF0000} Texto}!}

{\color[HTML]{00FF00} Texto. Se hace con \verb!{\color[HTML]{00FF00} Texto}!}

{\color[HTML]{0000FF} Texto. Se hace con \verb!{\color[HTML]{0000FF} Texto}!}

{\color[HTML]{FFFF00} Texto. Se hace con \verb!{\color[HTML]{FFFF00} Texto}!}

{\color[HTML]{FF00FF} Texto. Se hace con \verb!{\color[HTML]{FF00FF} Texto}!}

{\color[HTML]{00FFFF} Texto. Se hace con \verb!{\color[HTML]{00FFFF} Texto}!}

{\color[HTML]{FFFFFF} Texto. Se hace con \verb!{\color[HTML]{FFFFFF} Texto}!}