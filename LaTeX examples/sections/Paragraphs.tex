\chapter{Párrafos}

Un parrafo se puede empezar a escribir tal cual, sin poner comandos ni ostias. Y cuando se quiere terminar el párrafo, se le da a enter dos veces, para que quede un espacio blanco en los dos cuerpos de texto del LaTex.

Y se sigue uno nuevo debajo.
Si solo se da un enter y se escribe justo debajo ocurre esto:
Que lo pilla todo como un mismo párrafo (como es el caso).

Un modo de escribir líneas una debajo de otra \newline
Es con el comando \verb!\newline! \newline
Alternativamente, se puede usar \verb!\\! \\
Y funciona de la misma forma

\paragraph{}Otro modo de empezar un párrafo es escribiendo el comando \verb!\paragraph{}!, que en este caso genera un párrafo con sangría (el inicio del párrafo empieza un poco más adeltante que el desto del párrafo).

\paragraph{Título del párrafo}Si se rellenan los \verb!{}! del comando, como en el comando \verb!\paragraph{Título del párrafo}!, el párrafo pasa a tener un título delante. En este caso, el título no tiene sangría.

Destacar que si se crea un párrafo con \verb!\paragraph{}! (con o sin título), éste deja una distancia más grande con el texto anterior que si simplemente se le da a enter (comparar este párrafo con enter, con en anterior con el comando)