\chapter{Ecuaciones y símbolos}

Hay varios símbolos que no se pueden poner por defecto en un texto, como por ejemplo \#, \&, \$ o \%. Básicamente, son símbolos que están reservados para usos específicos, como el caso de \%, que se usa para comentar un trozo de texto.

Para poder utilizar estos símbolos, hay que usar \verb!\! delante del símbolo, como si fuera un comando. De este modo, se inserta el símbolo en el texto. Por ejemplo, \verb!\&! inserta \& en el texto

Hay otros símbolos, como \euro, \dag, \ss, o \copyright, que requieren utilizar un comando específico en vez de un carácter del teclado. Por ejemplo, \verb!\euro! inserta el símbolo \euro.

Aquí está una lista con símbolos que pueden ser de utilidad, y como escribirlos. Para ver más símbolos, consultar internet o preguntar en el grupo de slack

\$, con \verb!\$!

\%, con \verb!\%!

\_, con \verb!\_!

\}, con \verb!\}!

\{, con \verb!\{!

\&, con \verb!\&!

\#, con \verb!\#!

\P, con \verb!\P!

\S, con \verb!\S!

\copyright, con \verb!\copyright!

\dag, con \verb!\dag!

\dots, con \verb!\dots!

\pounds, con \verb!\pounds!

\ae, con \verb!\ae!

\AE, con \verb!\AE!

\o, con \verb!\o!

\O, con \verb!\O!

\ss, con \verb!\ss!

En LaTeX también se pueden insertar ecuaciones. Básicamente hay dos tipos. Se pueden insertar ecuaciones en una misma línea, o insertarlas aparte. Para insertar una ecuación en una misma línea, se utiliza el símbolo \$, de modo que poniendo \verb!$1+1=2$!, se puede obtener la siguiente ecuación $1+1=2$ en medio de un texto.

Para insertar una ecuación aparte, se utilizan los comandos \verb!\begin{equation}! y \verb!\end{equation}!, y se pone la ecuación en medio de los dos.

\begin{equation}
1+1=2
\end{equation}

LaTeX dispone de un montón de expresiones y comandos para poder escribir cualquier tipo de ecuación. Consultando: \url{https://en.wikibooks.org/wiki/LaTeX/Mathematics} se puede ver una amplia lista de todos los comandos para ecuaciones. Un ejemplo de ecuación más complicada es:

\begin{equation}
\left[\vec{U}_{\infty}+\sum_{j=1}^{N}\Gamma_{j}\vec{v}_{ij}\right]\cdot\vec{n}_{i}=\vec{U}_{\infty}\cdot\vec{n}_{i}+\sum_{j=1}^{N}\Gamma_{j}\left[\vec{v}_{ij}\cdot\vec{n}_{i}\right]=0
\end{equation}