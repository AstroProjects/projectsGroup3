\chapter{Ejemplos de tablas}

A continuación están algunos ejemplos de tablas, ya centradas, con descripción y etiqueta. Entrando en el archivo \textit{TablesExamples.tex} que está en la carpeta \textit{sections} se puede copiar el código de cada tabla para utilizar donde se quiera.

Cada tabla vendrá con un título, un texto introductorio, y la tabla en sí.

\section*{Tabla estándar con ancho automático y columnas centrada}

La siguiente tabla tiene el formato estándar de líneas. Las columnas tienen ancho automático. El texto está centrado.

\begin{table}[h]
	\centering
	\begin{tabular}{|c|c|c|}
		\hline
		Texto 1 & Tarea 1 & Problema 1 \\
		\hline
		Texto 2 & Tarea 2 & Problema 2 \\
		\hline
		Texto 3 & Tarea 3 & Problema 3, el cuál es un problema un poco largo \\
		\hline
		Texto 4 & Tarea 4 & Problema 4 \\
		\hline
	\end{tabular}
	\caption[Tabla estándar con ancho automático y columnas centradas]{Esta tabla tiene el formato estándar de líneas. Las columnas tienen ancho automático. El texto está centrado.}
	\label{table1}
\end{table}

\section*{Tabla estándar con ancho automático y columnas alineadas a la izquierda}

La siguiente tabla tiene el formato estándar de líneas. Las columnas tienen ancho automático. El texto está alineado a la izquierda.

\begin{table}[h]
	\centering
	\begin{tabular}{|l|l|l|}
		\hline
		Texto 1 & Tarea 1 & Problema 1 \\
		\hline
		Texto 2 & Tarea 2 & Problema 2 \\
		\hline
		Texto 3 & Tarea 3 & Problema 3, el cuál es un problema un poco largo \\
		\hline
		Texto 4 & Tarea 4 & Problema 4 \\
		\hline
	\end{tabular}
	\caption[Tabla estándar con ancho automático y columnas alineadas a la izquierda]{Esta tabla tiene el formato estándar de líneas. Las columnas tienen ancho automático. El texto está alineado a la izquierda.}
	\label{table2}
\end{table}

\section*{Tabla estándar con ancho automático y columnas alineadas a la izquierda excepto la última alineada a la derecha}

La siguiente tabla tiene el formato estándar de líneas. Las columnas tienen ancho automático. El texto está alineado a la izquierda excepto en la última columna que está alineado a la derecha.

\begin{table}[h]
	\centering
	\begin{tabular}{|l|l|r|}
		\hline
		Texto 1 & Tarea 1 & Problema 1 \\
		\hline
		Texto 2 & Tarea 2 & Problema 2 \\
		\hline
		Texto 3 & Tarea 3 & Problema 3, el cuál es un problema un poco largo \\
		\hline
		Texto 4 & Tarea 4 & Problema 4 \\
		\hline
	\end{tabular}
	\caption[Tabla estándar con ancho automático y columnas alineadas a la izquierda excepto la última alineada a la derecha]{Esta tabla tiene el formato estándar de líneas. Las columnas tienen ancho automático. El texto está alineado a la izquierda excepto en la última columna que está alineado a la derecha.}
	\label{table3}
\end{table}

\section*{Tabla estándar con ancho fijo y columnas justificadas}

La siguiente tabla tiene el formato estándar de líneas. Las columnas tienen ancho fijo. El texto está justificado.

\begin{table}[h]
	\centering
	\begin{tabular}{|p{2cm}|p{2.5cm}|p{5cm}|}
		\hline
		Texto 1 & Tarea 1 & Problema 1 \\
		\hline
		Texto 2 & Tarea 2 & Problema 2 \\
		\hline
		Texto 3 & Tarea 3 & Problema 3, el cuál es un problema un poco largo \\
		\hline
		Texto 4 & Tarea 4 & Problema 4 \\
		\hline
	\end{tabular}
	\caption[Tabla estándar con ancho fijo y columnas justificadas]{Esta tabla tiene el formato estándar de líneas. Las columnas tienen ancho fijo. El texto está justificado.}
	\label{table4}
\end{table}

\section*{Tabla estándar con ancho fijo y columnas alineadas a la izquierda}

La siguiente tabla tiene el formato estándar de líneas. Las columnas tienen ancho fijo. El texto está alineado a la izquierda.

\begin{table}[h]
	\centering
	\begin{tabular}{|>{\raggedright\arraybackslash}p{2cm}|>{\raggedright\arraybackslash}p{2.5cm}|>{\raggedright\arraybackslash}p{5cm}|}
		\hline
		Texto 1 & Tarea 1 & Problema 1 \\
		\hline
		Texto 2 & Tarea 2 & Problema 2 \\
		\hline
		Texto 3 & Tarea 3 & Problema 3, el cuál es un problema un poco largo \\
		\hline
		Texto 4 & Tarea 4 & Problema 4 \\
		\hline
	\end{tabular}
	\caption[Tabla estándar con ancho fijo y columnas alineadas a la izquierda]{Esta tabla tiene el formato estándar de líneas. Las columnas tienen ancho fijo. El texto está alineado a la izquierda.}
	\label{table5}
\end{table}

\section*{Tabla estándar con ancho fijo y columnas alineadas a la izquierda excepto la última alineada a la derecha}

La siguiente tabla tiene el formato estándar de líneas. Las columnas tienen ancho fijo. El texto está alineado a la izquierda excepto en la última columna que está alineado a la derecha.

\begin{table}[h]
	\centering
	\begin{tabular}{|>{\raggedright\arraybackslash}p{2cm}|>{\raggedright\arraybackslash}p{2.5cm}|>{\raggedleft\arraybackslash}p{5cm}|}
		\hline
		Texto 1 & Tarea 1 & Problema 1 \\
		\hline
		Texto 2 & Tarea 2 & Problema 2 \\
		\hline
		Texto 3 & Tarea 3 & Problema 3, el cuál es un problema un poco largo \\
		\hline
		Texto 4 & Tarea 4 & Problema 4 \\
		\hline
	\end{tabular}
	\caption[Tabla estándar con ancho fijo y columnas alineadas a la izquierda excepto la última alineada a la derecha]{Esta tabla tiene el formato estándar de líneas. Las columnas tienen ancho fijo. El texto está alineado a la izquierda excepto en la última columna que está alineado a la derecha.}
	\label{table6}
\end{table}

\section*{Tabla con nuevo formato con ancho automático y columnas centrada}

La siguiente tabla tiene el nuevo formato de líneas. Las columnas tienen ancho automático. El texto está centrado.

\begin{table}[h]
	\centering
	\begin{tabular}{c c c}
		\toprule[2pt]
		\textbf{Texto 1} & \textbf{Tarea 1} & \textbf{Problema 1} \\
		\midrule[1.5pt]
		Texto 2 & Tarea 2 & Problema 2 \\
		\hline
		Texto 3 & Tarea 3 & Problema 3, el cuál es un problema un poco largo \\
		\hline
		Texto 4 & Tarea 4 & Problema 4 \\
		\bottomrule[2pt]
	\end{tabular}
	\caption[Tabla con nuevo formato con ancho automático y columnas centradas]{Esta tabla tiene el nuevo formato de líneas. Las columnas tienen ancho automático. El texto está centrado.}
	\label{table7}
\end{table}

\section*{Tabla con nuevo formato con ancho automático y columnas alineadas a la izquierda}

La siguiente tabla tiene el nuevo formato de líneas. Las columnas tienen ancho automático. El texto está alineado a la izquierda.

\begin{table}[h]
	\centering
	\begin{tabular}{l l l}
		\toprule[2pt]
		\textbf{Texto 1} & \textbf{Tarea 1} & \textbf{Problema 1} \\
		\midrule[1.5pt]
		Texto 2 & Tarea 2 & Problema 2 \\
		\hline
		Texto 3 & Tarea 3 & Problema 3, el cuál es un problema un poco largo \\
		\hline
		Texto 4 & Tarea 4 & Problema 4 \\
		\bottomrule[2pt]
	\end{tabular}
	\caption[Tabla con nuevo formato con ancho automático y columnas alineadas a la izquierda]{Esta tabla tiene el nuevo formato de líneas. Las columnas tienen ancho automático. El texto está alineado a la izquierda.}
	\label{table8}
\end{table}

\section*{Tabla con nuevo formato con ancho automático y columnas alineadas a la izquierda excepto la última alineada a la derecha}

La siguiente tabla tiene el nuevo formato de líneas. Las columnas tienen ancho automático. El texto está alineado a la izquierda excepto en la última columna que está alineado a la derecha.

\begin{table}[h]
	\centering
	\begin{tabular}{l l r}
		\toprule[2pt]
		\textbf{Texto 1} & \textbf{Tarea 1} & \textbf{Problema 1} \\
		\midrule[1.5pt]
		Texto 2 & Tarea 2 & Problema 2 \\
		\hline
		Texto 3 & Tarea 3 & Problema 3, el cuál es un problema un poco largo \\
		\hline
		Texto 4 & Tarea 4 & Problema 4 \\
		\bottomrule[2pt]
	\end{tabular}
	\caption[Tabla con nuevo formato con ancho automático y columnas alineadas a la izquierda excepto la última alineada a la derecha]{Esta tabla tiene el nuevo formato de líneas. Las columnas tienen ancho automático. El texto está alineado a la izquierda excepto en la última columna que está alineado a la derecha.}
	\label{table9}
\end{table}

\section*{Tabla con nuevo formato con ancho fijo y columnas justificadas}

La siguiente tabla tiene el nuevo formato de líneas. Las columnas tienen ancho fijo. El texto está justificado.

\begin{table}[h]
	\centering
	\begin{tabular}{p{2cm} p{2.5cm} p{5cm}}
		\toprule[2pt]
		\textbf{Texto 1} & \textbf{Tarea 1} & \textbf{Problema 1} \\
		\midrule[1.5pt]
		Texto 2 & Tarea 2 & Problema 2 \\
		\hline
		Texto 3 & Tarea 3 & Problema 3, el cuál es un problema un poco largo \\
		\hline
		Texto 4 & Tarea 4 & Problema 4 \\
		\bottomrule[2pt]
	\end{tabular}
	\caption[Tabla con nuevo formato con ancho fijo y columnas justificadas]{Esta tabla tiene el nuevo formato de líneas. Las columnas tienen ancho fijo. El texto está justificado.}
	\label{table10}
\end{table}

\section*{Tabla con nuevo formato con ancho fijo y columnas alineadas a la izquierda}

La siguiente tabla tiene el nuevo formato de líneas. Las columnas tienen ancho fijo. El texto está alineado a la izquierda.

\begin{table}[h]
	\centering
	\begin{tabular}{>{\raggedright\arraybackslash}p{2cm} >{\raggedright\arraybackslash}p{2.5cm} >{\raggedright\arraybackslash}p{5cm}}
		\toprule[2pt]
		\textbf{Texto 1} & \textbf{Tarea 1} & \textbf{Problema 1} \\
		\midrule[1.5pt]
		Texto 2 & Tarea 2 & Problema 2 \\
		\hline
		Texto 3 & Tarea 3 & Problema 3, el cuál es un problema un poco largo \\
		\hline
		Texto 4 & Tarea 4 & Problema 4 \\
		\bottomrule[2pt]
	\end{tabular}
	\caption[Tabla con nuevo formato con ancho fijo y columnas alineadas a la izquierda]{Esta tabla tiene el nuevo formato de líneas. Las columnas tienen ancho fijo. El texto está alineado a la izquierda.}
	\label{table11}
\end{table}

\section*{Tabla con nuevo formato con ancho fijo y columnas alineadas a la izquierda excepto la última alineada a la derecha}

La siguiente tabla tiene el nuevo formato de líneas. Las columnas tienen ancho fijo. El texto está alineado a la izquierda excepto en la última columna que está alineado a la derecha.

\begin{table}[h]
	\centering
	\begin{tabular}{>{\raggedright\arraybackslash}p{2cm} >{\raggedright\arraybackslash}p{2.5cm} >{\raggedleft\arraybackslash}p{5cm}}
		\toprule[2pt]
		\textbf{Texto 1} & \textbf{Tarea 1} & \textbf{Problema 1} \\
		\midrule[1.5pt]
		Texto 2 & Tarea 2 & Problema 2 \\
		\hline
		Texto 3 & Tarea 3 & Problema 3, el cuál es un problema un poco largo \\
		\hline
		Texto 4 & Tarea 4 & Problema 4 \\
		\bottomrule[2pt]
	\end{tabular}
	\caption[Tabla con nuevo formato con ancho fijo y columnas alineadas a la izquierda excepto la última alineada a la derecha]{Esta tabla tiene el nuevo formato de líneas. Las columnas tienen ancho fijo. El texto está alineado a la izquierda excepto en la última columna que está alineado a la derecha.}
	\label{table12}
\end{table}

\section*{Longtable con nuevo formato con ancho fijo y columnas justificadas}

La siguiente longtable tiene el nuevo formato de líneas. Las columnas tienen ancho fijo. El texto está justificado.

\begin{center}
	\begin{longtable}{p{2cm} p{2.5cm} p{5cm}}
		\toprule[2pt]
		\textbf{Texto 1} & \textbf{Tarea 1} & \textbf{Problema 1} \\
		\midrule[1.5pt] \endhead
		Texto 2 & Tarea 2 & Problema 2 \\
		\hline
		Texto 3 & Tarea 3 & Problema 3, el cuál es un problema un poco largo \\
		\hline
		Texto 4 & Tarea 4 & Problema 4 \\
		\bottomrule[2pt]
		\caption[Longtable con nuevo formato con ancho fijo y columnas justificadas]{Esta longtable tiene el nuevo formato de líneas. Las columnas tienen ancho fijo. El texto está justificado.}
		\label{table13}
	\end{longtable}
\end{center}

\section*{Longtable con nuevo formato con ancho fijo y columnas alineadas a la izquierda}

La siguiente longtable tiene el nuevo formato de líneas. Las columnas tienen ancho fijo. El texto está alineado a la izquierda.

\begin{center}
	\begin{longtable}{>{\raggedright\arraybackslash}p{2cm} >{\raggedright\arraybackslash}p{2.5cm} >{\raggedright\arraybackslash}p{5cm}}
		\toprule[2pt]
		\textbf{Texto 1} & \textbf{Tarea 1} & \textbf{Problema 1} \\
		\midrule[1.5pt] \endhead
		Texto 2 & Tarea 2 & Problema 2 \\
		\hline
		Texto 3 & Tarea 3 & Problema 3, el cuál es un problema un poco largo \\
		\hline
		Texto 4 & Tarea 4 & Problema 4 \\
		\bottomrule[2pt]
		\caption[Longtable con formato con ancho fijo y columnas alineadas a la izquierda]{Esta longtable tiene el nuevo formato de líneas. Las columnas tienen ancho fijo. El texto está alineado a la izquierda.}
		\label{table14}
	\end{longtable}
\end{center}

\section*{Longtable con nuevo formato con ancho fijo y columnas alineadas a la izquierda excepto la última alineada a la derecha}

La siguiente longtable tiene el nuevo formato de líneas. Las columnas tienen ancho fijo. El texto está alineado a la izquierda excepto en la última columna que está alineado a la derecha.

\begin{center}
	\begin{longtable}{>{\raggedright\arraybackslash}p{2cm} >{\raggedright\arraybackslash}p{2.5cm} >{\raggedleft\arraybackslash}p{5cm}}
		\toprule[2pt]
		\textbf{Texto 1} & \textbf{Tarea 1} & \textbf{Problema 1} \\
		\midrule[1.5pt] \endhead
		Texto 2 & Tarea 2 & Problema 2 \\
		\hline
		Texto 3 & Tarea 3 & Problema 3, el cuál es un problema un poco largo \\
		\hline
		Texto 4 & Tarea 4 & Problema 4 \\
		\bottomrule[2pt]
		\caption[Longtable con formato con ancho fijo y columnas alineadas a la izquierda excepto la última alineada a la derecha]{Esta longtable tiene el nuevo formato de líneas. Las columnas tienen ancho fijo. El texto está alineado a la izquierda excepto en la última columna que está alineado a la derecha.}
		\label{table15}
	\end{longtable}
\end{center}

\section*{Tabla estándar con ancho fijo y columnas alineadas a la izquierda y distintas líneas por celda}

La siguiente tabla tiene el formato estándar de líneas. Las columnas tienen ancho fijo. El texto está alineado a la izquierda. Hay celdas con varias líneas

\begin{table}[h]
	\centering
	\begin{tabular}{|>{\raggedright\arraybackslash}p{2cm}|>{\raggedright\arraybackslash}p{2.5cm}|>{\raggedright\arraybackslash}p{5cm}|}
		\hline
		Texto 1 & Tarea 1 & Problema 1 \\
		\hline
		Texto 2 & Tarea 2 & Problema 2.1 \newline Problema 2.2 \\
		\hline
		Texto 3 & Tarea 3.1 \newline Tarea 3.2 \newline Tarea 3.3 & Problema 3, el cuál es un problema un poco largo \\
		\hline
		Texto 4 & Tarea 4.1 \newline Tarea 4.2 & Problema 4.1 \newline Problema 4.2 \newline Problema 4.3 \\
		\hline
	\end{tabular}
	\caption[Tabla estándar con ancho fijo y columnas alineadas a la izquierda y distintas líneas por celda]{Esta tabla tiene el formato estándar de líneas. Las columnas tienen ancho fijo. El texto está alineado a la izquierda. Hay celdas con varias líneas}
	\label{table16}
\end{table}

\section*{Tabla con nuevo formato con ancho fijo y columnas alineadas a la izquierda y distintas líneas por celda}

La siguiente tabla tiene el nuevo formato de líneas. Las columnas tienen ancho fijo. El texto está alineado a la izquierda. Hay celdas con varias líneas

\begin{table}[h]
	\centering
	\begin{tabular}{>{\raggedright\arraybackslash}p{2cm} >{\raggedright\arraybackslash}p{2.5cm} >{\raggedright\arraybackslash}p{5cm}}
		\toprule[2pt]
		\textbf{Texto 1} & \textbf{Tarea 1} & \textbf{Problema 1} \\
		\midrule[1.5pt]
		Texto 2 & Tarea 2 & Problema 2.1 \newline Problema 2.2 \\
		\hline
		Texto 3 & Tarea 3.1 \newline Tarea 3.2 \newline Tarea 3.3 & Problema 3, el cuál es un problema un poco largo \\
		\hline
		Texto 4 & Tarea 4.1 \newline Tarea 4.2 & Problema 4.1 \newline Problema 4.2 \newline Problema 4.3 \\
		\bottomrule[2pt]
	\end{tabular}
	\caption[Tabla con nuevo formato con ancho fijo y columnas alineadas a la izquierda y distintas líneas por celda]{Esta tabla tiene el nuevo formato de líneas. Las columnas tienen ancho fijo. El texto está alineado a la izquierda. Hay celdas con varias líneas}
	\label{table17}
\end{table}

\section*{Longtable con nuevo formato con ancho fijo y columnas alineadas a la izquierda y distintas líneas por celda}

La siguiente longtable tiene el nuevo formato de líneas. Las columnas tienen ancho fijo. El texto está alineado a la izquierda. Hay celdas con varias líneas

\begin{center}
	\begin{longtable}{>{\raggedright\arraybackslash}p{2cm} >{\raggedright\arraybackslash}p{2.5cm} >{\raggedright\arraybackslash}p{5cm}}
		\toprule[2pt]
		\textbf{Texto 1} & \textbf{Tarea 1} & \textbf{Problema 1} \\
		\midrule[1.5pt] \endhead
		Texto 2 & Tarea 2 & Problema 2.1 \newline Problema 2.2 \\
		\hline
		Texto 3 & Tarea 3.1 \newline Tarea 3.2 \newline Tarea 3.3 & Problema 3, el cuál es un problema un poco largo \\
		\hline
		Texto 4 & Tarea 4.1 \newline Tarea 4.2 & Problema 4.1 \newline Problema 4.2 \newline Problema 4.3 \\
		\bottomrule[2pt]
		\caption[Longtable con nuevo formato con ancho fijo y columnas alineadas a la izquierda y distintas líneas por celda]{Esta longtable tiene el nuevo formato de líneas. Las columnas tienen ancho fijo. El texto está alineado a la izquierda. Hay celdas con varias líneas}
		\label{table18}
	\end{longtable}
\end{center}

\section*{Tabla estándar con ancho fijo y columnas centradas y celdas fusionadas}

La siguiente tabla tiene el formato estándar de líneas. Las columnas tienen ancho fijo. El texto está centrado. Hay celdas fusionadas

\begin{table}[h]
	\centering
	\begin{tabular}{|>{\centering\arraybackslash}p{2cm}|>{\centering\arraybackslash}p{2.5cm}|>{\centering\arraybackslash}p{5cm}|}
		\hline
		Texto 1 & Tarea 1 & Problema 1 \\
		\hline
		Texto 2 & Tarea 2 & Problema 2 \\ 
		\hline
		Texto 3 & \multicolumn{2}{c|}{Tarea y problema 3} \\
		\hline
		\multicolumn{3}{|c|}{Texto, tarea y problema 4} \\
		\hline
	\end{tabular}
	\caption[Tabla estándar con ancho fijo y columnas centradas y celdas fusionadas]{Esta tabla tiene el formato estándar de líneas. Las columnas tienen ancho fijo. El texto está centrado. Hay celdas fusionadas}
	\label{table19}
\end{table}

\section*{Tabla con nuevo formato con ancho fijo y columnas centradas y celdas fusionadas}

La siguiente tabla tiene el nuevo formato de líneas. Las columnas tienen ancho fijo. El texto está alineado a la izquierda. Hay celdas fusionadas

\begin{table}[h]
	\centering
	\begin{tabular}{>{\centering\arraybackslash}p{2cm} >{\centering\arraybackslash}p{2.5cm} >{\centering\arraybackslash}p{5cm}}
		\toprule[2pt]
		\textbf{Texto 1} & \textbf{Tarea 1} & \textbf{Problema 1} \\
		\midrule[1.5pt]
		Texto 2 & Tarea 2 & Problema 2 \\
		\hline
		Texto 3 & \multicolumn{2}{c}{Tarea y problema 3} \\
		\hline
		\multicolumn{3}{c}{Texto, tarea y problema 4} \\
		\bottomrule[2pt]
	\end{tabular}
	\caption[Tabla con nuevo formato con ancho fijo y columnas centradas y celdas fusionadas]{Esta tabla tiene el nuevo formato de líneas. Las columnas tienen ancho fijo. El texto está alineado a la izquierda. Hay celdas fusionadas}
	\label{table20}
\end{table}