\section{Acceptance Criteria}

The acceptance criteria are important to define the performance requirements and the essential conditions that the deliverables of the project must attain. It is a quality parameter and their fulfilment indicates that the client's needs have been accomplished.

\begin{longtable}[H]{ p{4cm} p{10cm} }
	
	\toprule[2pt]
	
	\textbf{Item} & \textbf{Description} \\ 
	
	\midrule[1.5pt] \endhead
	
	Research and innovation & The project must be ambitious and use all available resources to obtain the best result. In this way, it must include the most appropriate technology that there is so far and, if it is in the development phase, add a section of research. \\ 
	
	\hline
	
	Quality & The content of the project documentation must be clear, complete and understandable. Furthermore, it must be well structured, dividing the information into approach, development and conclusions. \newline
	All the documentation included in the project must first pass through an inspection of the quality department. \\ 
	
	\hline
	
	Test and validations & The evaluation and validation tests must be carried out periodically and be registered in the project documentation, in such a way that there is a record of the different versions of the application throughout the development. \newline
	The information of these tests must be presented clearly and refer to the regulations concerned, in addition to be verifiable. \newline
	The results of these tests should be used to analyze the service level of the application and improve on later versions. \\
	
	\hline
	
	Technical documents & The application must have a user manual both internally and externally and attach the necessary information for its development. \newline
	The performance of the final product must be reflected in a data sheet, it must also include in the documentation the datasheet of the different components that are part of the application. \\
	
	\hline
	
	Viability & The project must be viable economically and technically, so that its realization is possible. \newline
	The different parts of the project must be submitted at the individual level to a study that checks if it is possible to do them and, if not, search for an alternative. \newline
	The budget of the project must comply with the financial requirements of the European Union. Hence, a balance is to be made to ensure that the allowed limit is not exceeded. \\
	
	\hline
	
	Performance & The systems used in the project must be able to guarantee the right functioning of the application. An important aspect of the project is its performance, in this way, as it progresses, it aims to increase the efficiency and quantify this increase in the different phases. \\
	
	\hline
	
	Collaboration & It is interesting to obtain a better result to collaborate with legal entities from different countries, like universities and research groups. Moreover, some collaborations with SMEs should be tried, so that they can benefit and grow in the market.	\\
	
	\hline
	
	Transparency & In case information about the project is required by the part of official organisations of the European Union or by the different stakeholders that participate in it, transparency has to be considered when sharing information. \\
	
	\hline
	
	Legal requirements & The applications and products of this project must have, if required, the certification and approval of the different legislative and ethical frameworks. \\
	
	\bottomrule[2pt]
	
	\caption{Acceptance criteria}
\end{longtable}