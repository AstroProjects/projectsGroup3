\chapter{Project Charter}

\section{Project Purpose and Justification}

\paragraph{} 

Since the first Earth observation (EO) satellite was launched in 1957, the need to gather remote sensed information about planet Earth has been increasing along with its technology. Today, after 60 years, EO has become a key piece of society by providing data for maritime, weather and air quality control together with urban development. 

Moreover, modern civilizations are now wanted and required to continue to be developed in sustainable ways and its negative impacts to be controlled and minimized. Is in this area where EO plays a significant role being able to collect data to give awareness as well as to provide information for social well-being and sustainable improvements.

On the other hand, besides the large amount of gathered data and the sophisticated technology used, in the recent years there has been and increasing demand for EO improved technology that allows going further in terms of reliability, size, resolution, efficiency and accuracy along with improved data processing systems with better combined data reliance and capable of give information for a a higher number of applications. 


Hence, this project aims to research and improve the existing EO technologies for remote sensing, develop a data processing software along with it containing machine learning algorithms focused on urban sustainable developments such as pollution and gas emission control, traffic monitoring, weather prediction, management of urban areas, regional and local planning, tourism development and cityscapes designs, and develop a web based for data sharing. 


The accomplishment of the project will demonstrate significant knowledge and enhancements concerning reliability, size, resolution, efficiency and accuracy among others of the current remote sensing technologies that not only will allow to gather better and more specific EO data, improving the results on their application fields but it will suppose a step forward in all those areas involving remote sensing from which the European society will benefit. 

Also, the implemented data processor will provide information sets about sustainable development issues such as geospatial indicators, pollution levels or gas emissions that will benefit companies and initiatives from world-wide and local organisations to carry out social and green actions and will support the United Nation projects: UN 2030 Agenda for Sustainable Development and The Paris Agreement on Climate Change. Furthermore the project sharing web will allow the public to interact enriching and contributing in the integration of space in economy and society.
 
Additionally, the attainment of the improved sensors and data processing software is expected to serve process the data gathered by the Sentinels' satellites in order to benefit the current on-going Copernicus programme missions so as to equip them with better remote sensing technologies in the near future.  


\subsection{Vision}
We are committed to achieving substantial improvements in state-of-the-art EO technologies such as radar and optical systems, leading to a strengthening of Europe's position and competitiveness in this field. 



\subsection{Objectives}
\paragraph{}The aim of this topic is to demonstrate, in a relevant environment, technologies, systems and sub-systems for EO. Proposals should demonstrate significant improvements in such areas as miniaturisation, power reduction, efficiency, versatility, and/or increased functionality, and should demonstrate at the viable extent complementarity to activities already funded by Member States and the European Space Agency. Proposals should also ensure system readiness for operational services and provide leverage on industry competitiveness, particularly on export markets.

The key OBJECTIVES for this project are:
\begin{enumerate}
	\item Tal i pasqual
\end{enumerate}

\subsection{Scope}


The SCOPE for this project is:

{\bfseries Engineering} 
\begin{itemize}

	\item Research and analysis of the current space applications and requirements of the following optical and radar systems:

		\begin{itemize}

		\item LIDAR
		\item Radar
		\item Super-spectral
		\item Hyperspectral
		\item Limb sounders
		\item Gravimetry
		\item High quantum efficiency photodectectos
		\item High precisition optical beam scanning and pointing
		\item Advanced infrared technologies

		\end{itemize}

	\item Research of the contributions of current space technologies to urban development.

	\item Selection of the most promising systems to profit Earth Observation to air composition and terrain analysis.
	
	\item Development of sensor's preliminary design defining the minimum performance parameters in order to improve the existing technologies.

	\item ¿¿¿Interaction platform ???

	\item Development of a mock-up by following the preliminary design.

	\item Testing and validation of the prototype in a space simulated environment.

	\item Design closure and development of the product.

\end{itemize}

{\bfseries Business planning and exploitation of results}
\begin{itemize}
	\item Market analysis and of the potential suppliers and selection of these.
	\item Market analysis of the potential costumers.
	\item Elaboration of a business plan. 

\end{itemize}

\textbf{Quality} \textit{esto no acaba de parecer apropiado en el scope}
\begin{itemize}
	\item Every document goes through four stages to be approved:
	\begin{itemize}
		\item{-} Guidelines preparation
		\item{-} Document revision
		\item{-} Document rectification
		\item{-} Document approval
	\end{itemize}
	\item Elaboration of periodic reports in order to have continuous control over the development of the project.

\end{itemize}

\textbf{Communication and dissemination strategies} \textit{esto no acaba de parecer apropiado en el scope}

\begin{itemize}
	\item Implementation of a dissemination plan to announce the product combining online and offline dissemination.	
	 \item Development of 
	  a website. 
	  \item Use of social media marketing power. 
	  \item Conduct several conferences.
	  \item  Let to know the product improvements through technology demonstrators.
\end{itemize}








