\subsection{Scope}
\paragraph{}Each proposal shall address only one of the following subtopics:

a) Very high resolution optical EO for LEO and/or high resolution optical EO for GEO/HEO instrument technologies, with focus on improving payload (e.g. radiometric and spectral parameters, spatial resolution, swath), including detectors, materials and solutions for stable and large optomechanical elements and systems (e.g. lightweight telescope mirrors with metre-level diameter) focal planes, wave front error and line of sight control, high performance actuators, multispectral filters for large focal plane;.

b) Competitive remote sensing instruments and space systems: innovations supporting readiness advancements for next generation systems in the optical and radio frequency domains (active/passive), technologies enabling advanced system solutions (including small satellites possibly in convoy with 

existing space assets), on-board image processing and detectors for video imaging with increased swath and resolution, technologies for super- and hyperspectral imaging instruments with high performance, radio occultation sensors, low cost high resolution telescopes and radar imaging systems;

c) Disruptive technologies for remote sensing, as technology building blocks for innovative LiDAR (Light Detection And Ranging) and radar instruments (including cost-effective wide-swath altimetry and imaging systems), super-spectral and hyperspectral payloads with wide spectral and/or coverage, limb sounders and gravimetry payloads; high quantum efficiency photo detectors and high-precision optical beam scanning and pointing; advanced infrared (IR) technologies (optical filters, detectors and electronics);

d) On-board data processing: integrated multi-instrument on-board payload data processing for resource-constrained missions; solutions for high observation reactivity and real-time applications such as very high performance payload processing; on board data/image optimisation and compression for advanced video and image pre-processing as well as smart on-board data/image analysis; data flow optimisation for new missions, including impacts on the evolution of associated ground segment, for enhancement of overall processing power and speed over the full chain and for supporting massive data processing and machine learning in EO applications;

e) Advanced SAR/Radar technologies: step up maturity in new sensing concepts and technologies such as large and active antennas and reflectors, including multi-frequency concepts; enablers for digital beam-forming and beam-hopping interferometric systems, and for other concepts, such as large swath maritime surveillance radar, active sensing/processing of SAR ships, data fusion integration with new generation Automatic Identification Systems (AIS);

Low cost solutions based on components off the shelf (COTS) are encouraged. Participation of industry, in particular SMEs, is encouraged. Activities shall be complementary and create synergy with other European activities in the same domain.

To this end, proposals shall include the following tasks:
\begin{itemize}
	\item Analysis of relevant available roadmaps, including roadmaps developed in the context of actions for the development of Key Enabling Technologies supported by the Union;
	\item Commercial assessment of the supply chain technology in the space or non-space domains and, if applicable, a business plan for commercialisation with a full range (preload) of recurring products.
\end{itemize}

The Commission considers that proposals requesting a contribution from the EU of between EUR 2 and 3 million would allow this specific challenge to be addressed appropriately. Nonetheless, this does not preclude submission and selection of proposals requesting other amounts.
