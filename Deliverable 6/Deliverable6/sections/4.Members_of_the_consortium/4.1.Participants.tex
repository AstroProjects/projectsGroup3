\section{Participants}

All the members of the consortium are listed in Table \ref{participants}, which states the name of the company, its short name, its country of origin and the type of business. Regarding the last column, the following abbreviations have been used:

\begin{itemize}
	\item SME: Small and medium-sized enterprise.
	\item MNC: Multinational corporation.
	\item R\&D: Research and development centre.
\end{itemize} 

\begin{table}[H]
	\centering
	\begin{tabular}{p{1cm} >{\raggedright\arraybackslash}p{5.7cm} p{2cm} p{2cm} p{2cm}}
		
		\toprule[2pt]
		
		\textbf{\#} & \textbf{Participant legal name} & \textbf{Short name} & \textbf{Country} & \textbf{Type}\\
		
		\midrule[1.5pt] 
		
		1 & Airbus Defence and Space GmbH & ADS & Germany & MNC \vspace{0.2cm}\\
		
		\midrule
		
		2 & BHO Legal Rechtsanwälte Partnership & BHO & Germany & MNC \vspace{0.2cm}\\
		
		\midrule
		
		3 & Deimos Space S.L.U. & DS & Spain & SME \vspace{0.2cm}\\
		
		\midrule
		
		4 & High Innovative Remote Observation (HIRO) & HR & Spain & SME \vspace{0.2cm}\\
		
		\midrule
		
		5 & ICUBE-SERTIT & IS & France & R\&D \vspace{0.2cm}\\
		
		\midrule
		
		6 & Remote Sensing Application Center (ReSAC) & RSAC & Bulgaria & R\&D \vspace{0.2cm}\\
		
		\midrule
		
		7 & Thales Alenia Space SAS & TAS & France & MNC \vspace{0.2cm}\\
		
		\midrule
		
		8 & VITO nv & VT & Belgium & R\&D \vspace{0.2cm}\\
		
		\bottomrule[2pt]
		
	\end{tabular}
	\caption{List of participants}
	\label{participants}
\end{table}



\begin{longtable}[H]{|p{0.7cm}|p{4cm}|p{7cm}|p{1.3cm}|}
	\hline
	\begin{center} Nº1 \end{center} & \begin{center} \includegraphics[scale=0.09]{./logos/Airbus-defence-and-space-logo} \end{center} & \begin{center} \textbf{Organisation name:} Airbus Defence and Space GmbH \newline \textbf{Website:} http://www.geo-airbusds.com \end{center} & \begin{center} Type: \newline MNC \end{center} \\ \hline
	
	\multicolumn{4}{|p{13cm}|}{\textbf{Overall description}}  \\ \hline
	
	\multicolumn{4}{|p{14.5cm}|}{Airbus Defence and Space is a division of Airbus, the largest aeronautics and space company in Europe. It is the world's second largest space company in the world. Their work comprises different departments: Earth Observation, Telecom Satellites, Human Spaceflight, Launchers, Satellite Navigation, Space Exploration, Space Equipment and Space Data Highway. Regarding Earth Observation, Airbus Defence and Space has built and delivered almost 50 satellite systems since 1986, accumulating over 300 years of in-orbit operation. None of these missions has ever failed in orbit. Some of their work includes meteorology satellites, sensing satellites and data intelligence services.}  \\ \hline
	
	\multicolumn{4}{|p{13cm}|}{\textbf{Role within the project}}   \\ \hline
	
	\multicolumn{4}{|p{14.5cm}|}{The main role of the company in the project is the research, development and production of the space systems, including the satellite sensors and systems.}  \\ \hline
	
	\multicolumn{4}{|p{14.5cm}|}{\textbf{Previous R\&D Experience relevant to the project}}  \\ \hline
	
	\multicolumn{4}{|p{13cm}|}{Airbus Defence and Space has over 25 years of experience of working with the European Space Agency, developing missions such as SMOS, CryoSat-2, GOCE and Swarm.}  \\ \hline
	
	\multicolumn{4}{|p{13cm}|}{\textbf{Key persons assigned to the project}}   \\ \hline
	
	
	\multicolumn{4}{|p{14.5cm}|}{Matthew Perren (male) received his Physics and Electronics degree from the Brunel University in 1986. He then studied a MSc in Satellite Communications at the University of Surrey in 1992. After working several years in other companies, he became head of Future Programmes, Earth Observation Navigation and Science Div. at Airbus Defence and Space in 2014.}  \\ \hline
	
	\multicolumn{4}{|p{13cm}|}{\textbf{Selected publications or products/services relevant to the project}}  \\ \hline
	
	\multicolumn{4}{|p{14.5cm}|}{-}  \\ \hline
	
	\multicolumn{4}{|p{13cm}|}{\textbf{Participation in relevant National or European research projects}}  \\ \hline
	
	\multicolumn{4}{|p{14.5cm}|}{Airbus is a known supplier of the European Space Agency. It also is a driving force of the project Copernicus, since it has developed and built the Sentinel Satellites and several of their instruments.}  \\ \hline
	
	\multicolumn{4}{|p{13cm}|}{\textbf{Equipment involved}}  \\ \hline
	
	\multicolumn{4}{|p{14.5cm}|}{Facilites of Airbus Defence and Space GmbH.}  \\ \hline
	\caption{Participant Nº1}
\end{longtable}



\begin{longtable}[H]{|p{0.7cm}|p{4cm}|p{7cm}|p{1.3cm}|}
	\hline
	\begin{center} Nº2 \end{center} & \begin{center} \includegraphics[scale=0.4]{./logos/BHO-logo} \end{center} & \begin{center} \textbf{Organisation name:} BHO Legal Rechtsanwälte Partnership \newline \textbf{Website:} http://www.bho-legal.com \end{center} & \begin{center} Type: \newline MNC \end{center} \\ \hline
	
	\multicolumn{4}{|p{13cm}|}{\textbf{Overall description}}  \\ \hline
	
	\multicolumn{4}{|p{14.5cm}|}{BHO Legal is a law firm with focus on aerospace and high-technology projects. Its areas of expertise are international space law, EU regulatory law, procurement, IT, data protection, R\&D and contract law. They are currently focused on satellite navigation, Earth Observation and related R\&D projects.}  \\ \hline
	
	\multicolumn{4}{|p{13cm}|}{\textbf{Role within the project}}   \\ \hline
	
	\multicolumn{4}{|p{14.5cm}|}{The main role of the company in the project is to provide legal and business advice to develop a business plan that takes into account intellectual property management, data protection and exploitation of the product.}  \\ \hline
	
	\multicolumn{4}{|p{13cm}|}{\textbf{Previous R\&D Experience relevant to the project}}  \\ \hline
	
	\multicolumn{4}{|p{14.5cm}|}{BHO Legal has worked as a legal advisor for projects related to satellite for communications, navigation and Earth Observation, as well as for different ESA projects and EU research programmes (for example, the Copernicus programme).}  \\ \hline
	
	\multicolumn{4}{|p{13cm}|}{\textbf{Key persons assigned to the project}}   \\ \hline
	
	\multicolumn{4}{|p{14.5cm}|}{Oliver Heinrich (male) specialises in the legal management of large projects. He also advises on questions concerning national, European and international procurement law. Among Oliver's clients are international and medium-sized companies mainly from the Aerospace, Telecommunications and Navigation Industry. Furthermore, he advises large research institutions. Oliver is Member of the Board of Directors of UVS International, Member of the Extended Board of UAV DACH where he is head of the association’s expert group "Air Law and Insurance" and head of its legal working group. He got his degree in Anglo-American Law from Trier University in 1997.}  \\ \hline
	
	\multicolumn{4}{|p{13cm}|}{\textbf{Selected publications or products/services relevant to the project}}  \\ \hline
	
	\multicolumn{4}{|p{14.5cm}|}{-}  \\ \hline
	
	\multicolumn{4}{|p{13cm}|}{\textbf{Participation in relevant National or European research projects}}  \\ \hline
	
	\multicolumn{4}{|p{14.5cm}|}{Prior to working as a full-time attorney, Oliver was a project manager for the European Satellite Navigation System Galileo at the German Aerospace Centre DLR. As authorised officer of TeleOp GmbH and legal advisor to the international consortium for the Galileo concession he gathered extensive practical experience in large international projects. }  \\ \hline
	
	\multicolumn{4}{|p{13cm}|}{\textbf{Equipment involved}}  \\ \hline
	
	\multicolumn{4}{|p{14.5cm}|}{Facilities of BHO Legal Rechtsanwälte Partnership.}  \\ \hline
	\caption{Participant Nº2}
\end{longtable}



\begin{longtable}[H]{|p{0.7cm}|p{4cm}|p{7cm}|p{1.3cm}|}
	\hline
	\begin{center} Nº3 \end{center} & \begin{center} \includegraphics[scale=0.05]{./logos/Blue-Logo_White-Background_EN} \end{center} & \begin{center} \textbf{Organisation name:} Deimos Space S.L.U. \newline \textbf{Website:} http://www.deimos-space.com/en/ \end{center} & \begin{center} Type: \newline SME \end{center} \\ \hline
	
	\multicolumn{4}{|p{13cm}|}{\textbf{Overall description}}  \\ \hline
	
	\multicolumn{4}{|p{14.5cm}|}{Deimos Space specialises in the design, engineering and development of solutions and systems integration in the aerospace, satellite systems, remote sensing, information systems and telecommunications network sectors. They are manufacturers, software developers, data providers and data processing experts.}  \\ \hline
	
	\multicolumn{4}{|p{13cm}|}{\textbf{Role within the project}}   \\ \hline
	
	\multicolumn{4}{|p{14.5cm}|}{The main role of the company in the project is the design and development of the space systems, including the satellite sensors and systems.}  \\ \hline
	
	\multicolumn{4}{|p{13cm}|}{\textbf{Previous R\&D Experience relevant to the project}}  \\ \hline
	
	\multicolumn{4}{|p{14.5cm}|}{Deimos Space coordinates different projects regarding Earth Observation: SIMOcean, NextGEOSS and EO-ALERT. In June 2014 they launched the Deimos-2, a satellite designed by them. It provides near-real time image tasking, downloading, processing and delivery to the end user.}  \\ \hline
	
	\multicolumn{4}{|p{13cm}|}{\textbf{Key persons assigned to the project}}   \\ \hline
	
	\multicolumn{4}{|p{14.5cm}|}{Ismael López (male) received his Telecommunications Engineering degree from the University of Sevilla in 2001. After working some years as a Project Engineer, he became Project Manager in Deimos in 2010. }  \\ \hline
	
	\multicolumn{4}{|p{13cm}|}{\textbf{Selected publications or products/services relevant to the project}}  \\ \hline
	
	\multicolumn{4}{|p{14.5cm}|}{
	\begin{itemize}
		\item \vspace{-0.5cm}Satellite Operations Strategies and Experience in DEIMOS-1 and DEIMOS-2 Missions.
		\item Collision Avoidance Operations of DEIMOS-1 and DEIMOS-2 Missions.\vspace{-0.3cm}
	\end{itemize}	}  \\ \hline
	
	\multicolumn{4}{|p{13cm}|}{\textbf{Participation in relevant National or European research projects}}  \\ \hline
	
	\multicolumn{4}{|p{14.5cm}|}{A subdivision of Deimos Space named Deimos Engenharia leads a Horizon 2020 Project called NextGEOSS. It proposes a centralised hub for Earth Observation data, where users can connect to access data and deploy EO-based applications. Deimos also coordinates another H2020 project named EO-ALERT, which aims to achieve very high throughput and very low latency in the delivery of Earth observation images, moving beyond the state of the art. Finally, they also collaborate with the Portuguese government in the SIMOcean project, which provides aims to create a national marine database.}  \\ \hline
	
	\multicolumn{4}{|p{13cm}|}{\textbf{Equipment involved}}  \\ \hline
	
	\multicolumn{4}{|p{14.5cm}|}{Facilities of Deimos Space S.L.U.}  \\ \hline
	\caption{Participant Nº3}
\end{longtable}



\begin{longtable}[H]{|p{0.7cm}|p{4cm}|p{7cm}|p{1.3cm}|}
	\hline
	\begin{center} Nº4 \end{center} & \begin{center} \includegraphics[scale=0.07]{./logos/logo} \end{center} & \begin{center} \textbf{Organisation name:} High Innovative Remote Observation \newline \textbf{Website:} http://www.hiro.com \end{center} & \begin{center} Type: \newline SME \end{center} \\ \hline
	
	\multicolumn{4}{|p{13cm}|}{\textbf{Overall description}}  \\ \hline
	
	\multicolumn{4}{|p{14.5cm}|}{HIRO is a European organization centred in the development of aerospace projects. Their main goal is to use space technologies to solve Earth problems. They plan to use Remote Sensing and Earth Observation data to develop sustainable plans for land use, natural disasters management, natural resources control, climatology, infrastructure and urban development, etc.}  \\ \hline
	
	\multicolumn{4}{|p{13cm}|}{\textbf{Role within the project}}   \\ \hline
	
	\multicolumn{4}{|p{14.5cm}|}{HIRO is going to perform the research in space technology and the design of the product. It is also the company's role to manage and administrate the project.}  \\ \hline
	
	\multicolumn{4}{|p{13cm}|}{\textbf{Previous R\&D Experience relevant to the project}}  \\ \hline
	
	\multicolumn{4}{|p{14.5cm}|}{-}  \\ \hline
	
	\multicolumn{4}{|p{13cm}|}{\textbf{Key persons assigned to the project}}   \\ \hline
	
	\multicolumn{4}{|p{14.5cm}|}{Pol Fontanes (male) studied Aerospace Engineering in the Polytechnic University of Catalonia and he is currently studying a MSc of Aerospace Engineering. He has experience as a researcher in the UPC Lightning Research Group and he is also a Integrated systems developer.} \\ \hline
	
	\multicolumn{4}{|p{13cm}|}{\textbf{Selected publications or products/services relevant to the project}}  \\ \hline
	
	\multicolumn{4}{|p{14.5cm}|}{
		\begin{itemize}
			\item \vspace{-0.5cm}Measurement of electric potential of isolated vertical conductive wires pulled by a drone under fair weather conditions.
			\item Validation of an Experimental Cloud Infrastructure for Earth Observation Services.
			\item Cloud Architecture for Processing and Distribution of Satellites Imagery.
			\item Validation of an experimental on-demand cloud infrastructure for Earth Observation Web Services.\vspace{-0.3cm}
		\end{itemize}}  \\ \hline
	
	\multicolumn{4}{|p{13cm}|}{\textbf{Participation in relevant National or European research projects}}  \\ \hline
	
	\multicolumn{4}{|p{14.5cm}|}{-}  \\ \hline
	
	\multicolumn{4}{|p{13cm}|}{\textbf{Equipment involved}}  \\ \hline
	
	\multicolumn{4}{|p{14.5cm}|}{Facilities of HIRO.}  \\ \hline
	\caption{Participant Nº4}
\end{longtable}



\begin{longtable}[H]{|p{0.7cm}|p{4cm}|p{7cm}|p{1.3cm}|}
	\hline
	\begin{center} Nº5 \end{center} & \begin{center} \includegraphics[scale=0.6]{./logos/Icube_web} \end{center} & \begin{center} \textbf{Organisation name:} ICUBE-SERTIT \newline \textbf{Website:} http://sertit.u-strasbg.fr/ \end{center} & \begin{center} Type: \newline R\&D \end{center} \\ \hline
	
	\multicolumn{4}{|p{13cm}|}{\textbf{Overall description}}  \\ \hline
	
	\multicolumn{4}{|p{14.5cm}|}{SERTIT (SErvice Régional de Traitement d'Image et de Télédétection) is a platform of the ICube laboratory that is part of the University of Strasburg. Their aim is to extract and format information from image data produced by Earth Observation systems, delivering products in natural resources monitoring, land management, urban planning, environmental survey and natural disaster and risk management. They refer to themselves as an intermediary between space research, digital technologies and operational user needs. They specialise in remote sensing R\&D activities.}  \\ \hline
	
	\multicolumn{4}{|p{13cm}|}{\textbf{Role within the project}}   \\ \hline
	
	\multicolumn{4}{|p{14.5cm}|}{The main role of the company is to provide advice in the application of data provided by EO satellites.}  \\ \hline
	
	\multicolumn{4}{|p{13cm}|}{\textbf{Previous R\&D Experience relevant to the project}}  \\ \hline
	
	\multicolumn{4}{|p{14.5cm}|}{SERTIT is a pioneer in the rapid mapping of natural disasters, covering hundreds of activations worldwide. They are currently working with ESA in the Sentinel satellite constellation in the development of land surface monitoring services with the data obtained by the satellites.}  \\ \hline
	
	\multicolumn{4}{|p{13cm}|}{\textbf{Key persons assigned to the project}}   \\ \hline
	
	\multicolumn{4}{|p{14.5cm}|}{Jean-François Rapp (male) got both his degree and his master from the University Louis Pasteur in Strasbourg. He works as Quality Manager of ICube, Research Engineer in a research team dedicated to methods and tools adapted to Innovative Design and Business Developer.  } \\ \hline
	
	\multicolumn{4}{|p{13cm}|}{\textbf{Selected publications or products/services relevant to the project}}  \\ \hline
	
	\multicolumn{4}{|p{14.5cm}|}{
		\begin{itemize} 
			\item \vspace{-0.5cm}Evaluation of Future Internet technologies for processing and distribution of satellite imagery.\vspace{-0.3cm}
		\end{itemize}}  \\ \hline
	
	\multicolumn{4}{|p{13cm}|}{\textbf{Participation in relevant National or European research projects}}  \\ \hline
	
	\multicolumn{4}{|p{14.5cm}|}{SERTIT is a service provider accompanying the French and ESA satellite development programs. It is currently involved in developing land surface monitoring services using ESA's Sentinel satellite constellation (Copernicus programme). They are also part of European Rapid Mapping (RM), a service that provides geospatial information to be used in emergency management activities.}  \\ \hline
	
	\multicolumn{4}{|p{13cm}|}{\textbf{Equipment involved}}  \\ \hline
	
	\multicolumn{4}{|p{14.5cm}|}{Facilities of ICUBE-SERTIT.}  \\ \hline
	\caption{Participant Nº5}
\end{longtable}



\begin{longtable}[H]{|p{0.7cm}|p{4cm}|p{7cm}|p{1.3cm}|}
	\hline
	\begin{center} Nº6 \end{center} & \begin{center} \includegraphics[scale=1.2]{./logos/logo_resac} \end{center} & \begin{center} \textbf{Organisation name:} Remote Sensing Application Center \newline \textbf{Website:} http://www.resac-bg.org/ \end{center} & \begin{center} Type:\newline R\&D \end{center} \\ \hline
	
	\multicolumn{4}{|p{13cm}|}{\textbf{Overall description}}  \\ \hline
	
	\multicolumn{4}{|p{14.5cm}|}{ReSAC is a non-profit organization that applies remote sensing in decision making for agricultural and environmental management, land use, soil and forestry inventory, water resources, environmental hazards and urban planning. It was part of the Bulgarian Aerospace Agency (BASA) until 2005. They offer services in distribution of satellite data, image processing and photogrammetric mapping and processing among others.}  \\ \hline
	
	\multicolumn{4}{|p{13cm}|}{\textbf{Role within the project}}   \\ \hline
	
	\multicolumn{4}{|p{14.5cm}|}{The main role of the company in the project is to provide advice in the application of the data obtained by EO satellites for its use in urban development.}  \\ \hline
	
	\multicolumn{4}{|p{13cm}|}{\textbf{Previous R\&D Experience relevant to the project}}  \\ \hline
	
	\multicolumn{4}{|p{14.5cm}|}{Since its foundation in 1998, ReSAC has participated in more than 40 projects, most of them related to the flood risk assessment or land cover mapping. For instance, in the recent years they have participated in the preparation of Floor Hazard Maps and Flood Risk Maps in different European areas. Regarding land study, for example, in 2012 they collaborated in the Romania-Bulgaria Cross Border Cooperation Programme, a project that designed a strategy for the sustainable spatial and economic development of the Romanian-Bulgarian Cross-Border area.}  \\ \hline
	
	\multicolumn{4}{|p{13cm}|}{\textbf{Key persons assigned to the project}}   \\ \hline
	
	\multicolumn{4}{|p{14.5cm}|}{Vessela Samoungi (female) is a physicist graduated from the University of Sofia. She also has a MSc in satellite Meteorology from that University. She has more than 8 years experience in EO at ReSAC and ASDE-Ecoregions NGOs and, currently she is a PhD student at SRTI-BAS in the field of SAR processing of PolInSAR in forestry studies.}  \\ \hline
	
	\multicolumn{4}{|p{13cm}|}{\textbf{Selected publications or products/services relevant to the project}}  \\ \hline
	
	\multicolumn{4}{|p{14.5cm}|}{
		\begin{itemize}
			\item  \vspace{-0.5cm}Monitoring of the risk of farmland abandonment as an efficient tool to assess the environmental and socio-economic impact of the Common Agriculture Policy
			\item High Nature Value farmland identification from satellite imagery, a comparison of two methodological approaches.\vspace{-0.3cm}
		\end{itemize}	
      }  \\ \hline
	
	\multicolumn{4}{|p{13cm}|}{\textbf{Participation in relevant National or European research projects}}  \\ \hline
	
	\multicolumn{4}{|p{14.5cm}|}{After leaving BASA, ReSAC has collaborated in many projects of the Bulgarian government, such as the Biodiversity and ecosystems programme (2015), in which did a mapping and assessment of the Bulgarian fresh water ecosystem services, or the quality analysis of the Land Parcel Identification System of Bulgaria (2013). They have also been part of European programmes, like Geoland2, a project that focused on land cover change, environmental stress and global vegetation monitoring in Europe.}  \\ \hline
	
	\multicolumn{4}{|p{13cm}|}{\textbf{Equipment involved}}  \\ \hline
	
	\multicolumn{4}{|p{14.5cm}|}{Facilities of ReSAC.}  \\ \hline
	\caption{Participant Nº6}
\end{longtable}


\pagebreak
\begin{longtable}[H]{|p{0.7cm}|p{4cm}|p{7cm}|p{1.3cm}|}
	\hline
	\begin{center} Nº7 \end{center} & \begin{center} \includegraphics[scale=0.25]{./logos/Thales_Alenia_Space_Belgium_logo} \end{center} & \begin{center} \textbf{Organisation name:} Thales Alenia Space SAS \newline \textbf{Website:} http://www.thalesgroup.com \end{center} & \begin{center} Type: \newline MNC \end{center} \\ \hline
	
	\multicolumn{4}{|p{13cm}|}{\textbf{Overall description}}  \\ \hline
	
	\multicolumn{4}{|p{14.5cm}|}{Thales Alenia Space SAS is Europe's largest satellite manufacturer. It designs, develops, integrates, tests, delivers and operates space systems worldwide. A joint venture between Thales and Leonardo, Thales Alenia Space offers services in telecommunications, earth observation, science and space exploration, navigation and orbital infrastructure and space transport. In the earth observation field, they specialise in high and very-high resolution optical and radar payloads, used in military and civilian applications: intelligence gathering, target designation, mapping, crisis management, meteorology, oceanography, climatology, etc.}  \\ \hline
	
	\multicolumn{4}{|p{13cm}|}{\textbf{Role within the project}}   \\ \hline
	
	\multicolumn{4}{|p{14.5cm}|}{The main role of the company in the project is the design, development, testing and integration of space systems.}  \\ \hline
	
	\multicolumn{4}{|p{13cm}|}{\textbf{Previous R\&D Experience relevant to the project}}  \\ \hline
	
	\multicolumn{4}{|p{14.5cm}|}{Thales Alenia Space is the exclusive supplier of all high and very-high resolution optical instruments for French intelligence satellites. They have also collaborated with space agencies. For example, building the Poseidon altimeters for the CNES/NASA Topex-Poseidon mission, which mapped the ocean surface topography; or developing the Siral very-high-resolution interferometry altimeter for ESA's CryoSat satellite.}  \\ \hline
	
	\multicolumn{4}{|p{13cm}|}{\textbf{Key persons assigned to the project}}   \\ \hline
	
	\multicolumn{4}{|p{14.5cm}|}{Philippe Keryer (male) got his degree of Electronic Engineer from the École Supérieure d'Électricité. As President of Networks group, leads a global, 20,000 people, multi-billion dollar telecoms organization and negociates with service providers, start-ups, and telecom leaders worldwide on products, services and intellectual property.}  \\ \hline
	
	\multicolumn{4}{|p{13cm}|}{\textbf{Selected publications or products/services relevant to the project}}  \\ \hline
	
	\multicolumn{4}{|p{14.5cm}|}{
	\begin{itemize}
		\item \vspace{-0.5cm}Testing Cloud Computing for Massive Space Data Processing, Storage and Distribution with Open-Source Geo-Software.\vspace{-0.3cm}
	\end{itemize}}  \\ \hline
	
	\multicolumn{4}{|p{13cm}|}{\textbf{Participation in relevant National or European research projects}}  \\ \hline
	
	\multicolumn{4}{|p{14.5cm}|}{Thales Alenia Space is part of the Copernicus programme, having developed the "water colour" instruments that are on board of Sentinel-3, MERIS and OLCI. They were also the prime contractors for the Egnos augmentation system, the precursor to Galileo.}  \\ \hline
	
	\multicolumn{4}{|p{13cm}|}{\textbf{Equipment involved}}  \\ \hline
	
	\multicolumn{4}{|p{14.5cm}|}{Facilities of Thales Alenia Space SAS.}  \\ \hline
	\caption{Participant Nº7}
\end{longtable}



\begin{longtable}[H]{|p{0.7cm}|p{4cm}|p{7cm}|p{1.3cm}|}
	\hline
	\begin{center} Nº8 \end{center} & \begin{center} \includegraphics[scale=0.35]{./logos/vito-logo} \end{center} & \begin{center} \textbf{Organisation name:} VITO nv \newline \textbf{Website:} https://vito.be/en/land-use \end{center} & \begin{center} Type: \newline R\&D \end{center} \\ \hline
	
	\multicolumn{4}{|p{13cm}|}{\textbf{Overall description}}  \\ \hline
	
	\multicolumn{4}{|p{14.5cm}|}{VITO is an independent Flemish research organisation in the area of sustainable development. They perform research and develop products in the fields of energy, environment and materials. In the environment field, VITO uses Remote Sensing and Earth Observation processes to perform studies on water, air and climate and land use. Regarding the use of land, they focus on 3D geological modelling, deep geothermal energy applications and land use policy analyses.}  \\ \hline
	
	\multicolumn{4}{|p{13cm}|}{\textbf{Role within the project}}   \\ \hline
	
	\multicolumn{4}{|p{14.5cm}|}{The main role of the company in the project is to provide advice in the use of remote sensing for urban development applications.}  \\ \hline
	
	\multicolumn{4}{|p{13cm}|}{\textbf{Previous R\&D Experience relevant to the project}}  \\ \hline
	
	\multicolumn{4}{|p{14.5cm}|}{In the recent years, VITO has collaborated in river basin management and infrastructure sustainable development projects in India. They also collaborate in ESA's Eagle Space program, a study that aims for the integration of space based capabilities such as EO satellites to help combat natural flooding and wildfire disasters.}  \\ \hline
	
	\multicolumn{4}{|p{13cm}|}{\textbf{Key persons assigned to the project}}   \\ \hline
	
	\multicolumn{4}{|p{14.5cm}|}{Steven Krekels (male) finished his studies of Telecommunications in De Nayer Instituut in 1995. Steven Krekels has been the manager of VITO’s Remote Sensing unit since November 2014. VITO Remote Sensing develops and operates space- and airborne-based earth observation systems that translate raw data into consumable information about population, growth, urban development, agriculture and vegetation, natural disasters, and more.}  \\ \hline
	
	\multicolumn{4}{|p{13cm}|}{\textbf{Selected publications or products/services relevant to the project}}  \\ \hline
	
	\multicolumn{4}{|p{14.5cm}|}{
	\begin{itemize}
		\item \vspace{-0.5cm}Toward Big Data in Green City
		\item Green Energy for a Green City—A Multi-Perspective Model Approach
		\item Advances in Sensors for Sustainable Smart Cities and Smart Buildings\vspace{-0.3cm}
	\end{itemize}}  \\ \hline

	\multicolumn{4}{|p{13cm}|}{\textbf{Participation in relevant National or European research projects}}  \\ \hline
	
	\multicolumn{4}{|p{14.5cm}|}{VITO collaborates in different ESA projects, such as Eagle Space or SSMART, a project that that wants to provide solutions for monitoring and managing transports of dangerous goods.}  \\ \hline
	
	\multicolumn{4}{|p{13cm}|}{\textbf{Equipment involved}}  \\ \hline
	
	\multicolumn{4}{|p{14.5cm}|}{Facilities of VITO nv.}  \\ \hline
	\caption{Participant Nº8}
\end{longtable}