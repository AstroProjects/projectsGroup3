\section{Management structure, milestones and procedures}

\subsection{Organisational Structure}

A complex organizational structure has been established given the complexity and scale of the project. On top of the organizational chain, a steering committee has been created and it will provide DEOS-UD with strategic command and solutions to problems that affect a significant part of the stakeholders in order to ensure a correct and efficient development of the project. Hand-to-hand with the steering committee, the advisory committee will provide the project leaders with tailored assistance in order to assure time and cost-efficient decisions are taken. There is also a business project team which will be in charge of assuring an economical resources correct management by providing careful tracing in the use of the budget along with a proper staff training in means of economic performance; the team is also ought to keep the steering committee updated with the latest information on earned value management parameters so that appropriate decisions are taken. Reinforcing the organizational structure of the project, a technical project team has been created as well in order to provide control over technical decisions in the project.
An organizational structure directed by a steering committee is specifically designed to fit such a large-scale and long-term project as this one is. The experience, capacities and diversity of the members it is composed by will play a key role in the outcome of the project while maintaining an efficient use of time and resources. The steering committee will take major decisions involving a significant fraction of the stakeholders. The business project team will be in charge of assessing the decisions involving budget modification or inter-department budget redistribution. Finally, decisions involving a modification or significant change in the technologies used during DEOS-UD progress will be in hands of the technical project team. Smaller affairs along with local inconveniences will be solved by the specific group affected by them. By having such a decision making distribution, DEOS-UD ensures a correct importance of the decision to experience ratio and thus assuring a more efficient time usage by providing every person within DEOS-UD with fitted responsibilities.
The milestones to accomplish are detailed in the following section; extracted from the third deliverable.      

\subsection{Acceptance Criteria and Milestones}

\textbf{Milestones: D2 apartado 1.3. Acceptance Criteria: D2 apartado 1.4 Poner toda la tabla.}

Based on the previous deliverables, the following criteria when stating a specific work package is established:
\begin{itemize} 	
	\item \textbf{WP1}: Management 
	\item \textbf{WP2}: Quality and administration
	\item \textbf{WP3}: State of the art
	\item \textbf{WP4}: Product development
	\item \textbf{WP5}: Simulation, testing and validation
	\item \textbf{WP6}: Business planning
	\item \textbf{WP7}: Communication and dissemination
	\item \textbf{WPx}: All Work Packages
\end{itemize}

The following table summarizes the milestones, including the work packages to which they are related, their due date and means of verification.

\begin{longtable}[H]{p{1.5cm} p{2.8cm} p{2.2cm} p{2cm} p{3.2cm} }
	\toprule[2pt]
	
	\textbf{Milestone No.} & \textbf{Milestone Name} & \textbf{Related WP} & \textbf{Due Date} & \textbf{Means of Verification} \\
	
	\midrule[1.5pt] 
	\endhead
	
	1 & Kick off meeting & WPx & 10/09/2018 & Agenda and Meeting Minutes \vspace{0.2cm} \\
	
	\midrule

	2 & Project management plan & WP1 & 05/10/2018 & Archived soft copy and management team evaluation meeting \vspace{0.2cm} \\
	
	\midrule
	
	3 & Business plan & WP6 & 05/10/2018 & Archived soft copy and business team evaluation meeting \vspace{0.2cm} \\

	\midrule

 	4 & Communication plan & WP7 & 05/10/2018 & Archived soft copy and marketing and communication team evaluation meeting \vspace{0.2cm} \\
 	 
 	\midrule
 	 
 	5 & State of the art completion & WP3 & 28/12/2018 & Archived soft copy and management and technical team evaluation meeting \vspace{0.2cm} \\
 	
 	\midrule
 
 	6 & Payload preliminary design & WP4 & 14/06/2019 & Technical documents review and technical team evaluation meeting \vspace{0.2cm} \\
 
 	\midrule
 
  	7 & Modular system preliminary design & WP4 & 06/09/2019 & Technical documents review and technical team evaluation meeting \vspace{0.2cm} \\
 
 	\midrule
 
 	8 & Interaction platform preliminary design & WP4 & 29/11/2019 & Technical documents review and technical team evaluation meeting \vspace{0.2cm} \\
 
 	\midrule
 
 	9 & Payload final design & WP4, WP2 & 12/06/2020 & Technical documents review and technical team evaluation meeting \vspace{0.2cm} \\
 
 	\midrule
 
 	10 & Modular system final design & WP4, WP2 & 04/09/2020 & Technical documents review and technical team evaluation meeting \vspace{0.2cm} \\
	
	\midrule

	11 & Interaction platform final design & WP4, WP2 & 27/11/2020 & Technical documents review and technical team evaluation meeting \vspace{0.2cm} \\

	\midrule

	12 & Prototype manufacturing & WP4 & 16/04/2021 & Prototype testing \vspace{0.2cm} \\

	\midrule

	13 & Individual systems testing & WP5 & 09/07/2021 & Systems testing \vspace{0.2cm} \\
	 
	\midrule

	14 & Full system testing & WP5 & 29/10/2021 & Full system testing \vspace{0.2cm} \\

	\midrule

	15 & Project completion & WPx & 21/01/2022 & Evaluation of all technical and non-technical documentation as well as prototype testing reports \vspace{0.2cm} \\

	\bottomrule[2pt]
	
	\caption{List of milestones}
	\label{workpackages}
\end{longtable}

\subsection{Quality Management}

The Quality management Plan defines the quality levels that must be achieved in order to accept the final product developed and the methods to ensure these levels. In this section the quality management plan is defined together with methods and tools to assure, control and improve it.
\section{Quality Assurance Approach}
One of the most important parts of the project is to ensure high quality levels in all its sections in order to provide a final product that meets the expectations of the possible future customers. In this section, the procedures and methods to ensure this high quality are detailed. 

At this point, it is important to recall high-level technical requirements defined previously in the Project Charter:
\begin{table}[H]
	\centering
	\begin{tabular}{l p{13.3cm}}
		
		\toprule[2pt]
		
		\textbf{Item} &  \textbf{Description}\\
		
		\midrule [1.5pt]
		
		T1 & Ensure the endurance of the overall system.\vspace{0.2cm}\\
		
		\midrule
		
		T2 & Readiness for operational services.\vspace{0.2cm}\\
		
		\midrule
		
		T3 & Ability to detect greenhouse gases.\vspace{0.2cm}\\
		
		\midrule
		
		T4 & Ability to detect weather patterns for proper weather forecasting applications.\vspace{0.2cm}\\
		
		\midrule
		
		T5 & Ability to perform a high precision terrain mapping for urban applications.\vspace{0.2cm}\\
		
		\midrule
		
		T6 & The system must have a program for automatic updates and self-revision of possible issues.\vspace{0.2cm}\\
		
		\midrule
		
		T7 & Availability of real-time information with a maximum delay of 1 second.\vspace{0.2cm}\\
		
		\midrule
		
		T8 & 15\% increase in the reliability and precision of the results compared to current technologies.\vspace{0.2cm}\\
		
		\bottomrule[2pt]
		
	\end{tabular}
	\caption{Technical requirements}
\end{table}

The quality assurance will be applied in the different steps of the project in order to obtain the desired results. These steps are:
\begin{itemize}
\item  Before manufacturing the prototype. Quality procedures must be applied over the final design to ensure it meets the requirements of the project.
\item During the manufacture. The procedures executed in the manufacture of the prototype must be validated to guarantee that they are suitable for the manufacture of the product.
\item Final product validation. The final product must be revised to ensure it fulfils the expected specifications. These validations will contain methods to check the quality of the software and the hardware of the project.
\end{itemize} 
Now that the quality needed has been specified, in the following sections the methods to control the quality and to improve the quality plan will be described. 

\subsubsection{Quality Control Approach}

The quality control approach of the project is divided in three main areas:
\begin{itemize}
	\item Documentation quality plan
	\item Technical quality plan
	\item Software quality plan
\end{itemize}

\textbf{Documentation quality plan}

All the documentation of the project has to follow a strict quality plan in order to ensure that no information is lost. To do so, there is a series of steps that have to be followed:
\begin{enumerate}
	\item Definition of the document
	\begin{itemize}
		\item Define the type of document and its content as well as the standards that it has to follow.
		\item Define the responsible of the document, the team that is going to work on it and the team that is going to verify it.
		\item Define the deadline for the document as well as any milestone that may be related to it.
	\end{itemize}
	\item Redaction of the document: There may be some periodic quality controls while the document is in progress to ensure that the quality plan is met.
	\item Review and approval: Once the document is finished, the responsible of that deliverable should perform the following tasks regarding the document:
	\begin{itemize}
		\item Grammatical revision.
		\item Consistency.
		\item References up to date.
		\item Check that the deliverable follows the acceptance criteria.
	\end{itemize}
	Then, the document can be delivered to the quality department. It will verify that the documentation follows the quality standards defined by the company. With the aim of guaranteeing a complete and trustful review, there should be at least two independent reviewers and they should not have been involved in the making of that document.
	If there is any review comment, it should be communicated to the deliverable responsible, since they have the final responsibility that the document meets the acceptance criteria.
\end{enumerate}
This documentation quality plan refers to the deliverables but also to the internal documents of the company.

\textbf{Technical quality plan}

Since part of the project consists of the design and construction of a prototype, it is necessary to ensure that it meets all the quality requirements to guarantee its proper operation. In order to do that, the following steps are defined:
\begin{enumerate}
	\item Definition of the quality plan: Before beginning with the design, a quality plan has to be done in order to define the acceptance criteria.
	\item Design: Once the plan is finished and the design phase starts, some quality controls have to be done periodically to guarantee that the design complies the requirements and follows the quality plan previously defined.
	\item Prototype and validation: During the construction of the prototype all the components and the production stages have to meet the acceptance criteria defined in the quality plan. Then, when the prototype is ready, a validation must be done in order to check that it fulfills all the requirements of the project as well as to verify that it complies the quality plan. This validation process has to follow the standards given by the industry.
\end{enumerate}

\textbf{Software quality plan}

The project not only consists of a prototype that should be constructed, but it also has a software that has to be verified. The following steps are defined to guarantee a satisfactory design of the implementation platform:
\begin{enumerate}
	\item Definition of the quality plan: Before starting with the coding, a software quality plan has to be defined. This document will set some standards that will have to be followed in the making of the interaction platform, such as coding and comment standards, to ensure a correct flow of information between the people who work on it as well as to avoid possible errors. It will also define the acceptance criteria that have to be met by the interaction platform.
	\item Coding phase: During the design phase, every modification of the code will have to be registered indicating the date and a description of the changes. Then, a review of the latest modifications should be done before making them definitive.	If an error is detected, it has to be immediately reported to the responsible of the software development. Then, an engineer will be assigned to solve it, and he/she will report it once the problem is solved.
	\item Implementation and validation: Once the interaction platform is operative, a validation has to be performed in order to ensure that it fulfils all the requirements of the project as well as to verify that it complies the software quality plan. This validation process has to follow the standards given by the industry.
\end{enumerate}
\subsubsection{Quality Improvement Approach}
Quality improvement (QI) is a formal analysis of practice performance and efforts done in order to improve the performance of the project with the main objective of increasing its efficiency. The information shown about QI models and tools has been extracted from \cite{aafp} and \cite{leansolutions}. A proper QI process requires some basics to succeed. These basics are the following ones:
\begin{itemize}
\item Establish a culture of quality in the project: Creation of QI teams, QI meetings, and QI goals.
\item Determine and prioritize potential areas of improvement: Define, according to the acceptance criteria of the project, the main areas of improvement.
\item Collect and analyse data: Determine the type of data to be collected and analyse it properly according to the project objectives.
\item Communication of results: Quality improvements should be transparent to the stakeholders in order to keep them satisfied. 
\end{itemize}
In this project, the six-sigma working philosophy will be implemented in order to improve quality. The objective of this philosophy is to adjust the existing processes in order to improve the quality and minimize the variability by reducing defects and irregularities. The model related to six-sigma philosophy that will be used is DMAIC. This model includes the following steps:
\begin{itemize}
\item Define: Set the objective of the problem or the existent defect. In this project, this definition will be done according to the acceptance criteria. The improvement of the quality plan is one of the objectives that will need to be taken into account.
\item Measurement: Measures are needed in order to have values for the problem or defect. In this project the measurements according to the effectiveness of the quality plan are:
\begin{itemize}
\item Number of iterations of a document/design to be approved.
\item Stakeholders satisfaction
\item Time needed to approve a document/design.
\item Number of defects detected by the quality department 
\end{itemize}
\item Analyse: Figure out the causes of the problem or defect and propose solutions.
\item Improve: Implement the approved solution.
\item Control: Control the implementation of the improvement, assure continuity and success.
\end{itemize}
\section{Quality Roles and Responsibilities}
In the following table, the quality roles for this project will be stated and its responsibilities defined. These roles are important so they will be the ones to control the implementation of the quality assurance, control, and improvement.

\begin{longtable}[H]{>{\raggedright\arraybackslash}p{5cm} p{9cm}}
	
	\toprule[2pt]
	
	\textbf{Role} &  \textbf{Responsibilities}\\
	
	\midrule [1.5pt]
	\endhead
	
	Project Manager & Final responsible for the quality of the project.\vspace{0.3cm}\newline Schedules meetings with the Quality Department in order to discuss the quality aspects of the project.\vspace{0.3cm}\newline Approves the quality plans of the project.\vspace{0.2cm} \\
	
	\midrule
		
	Project Manager Secretary & Helps the Project Manager in the tasks that he/she delegates.\vspace{0.2cm} \\
	
	\midrule
	
	Quality Manager & Main quality responsible of the project.\vspace{0.3cm}\newline Fixes the quality standards that all documents are required to fulfil.\vspace{0.3cm}\newline Reviews all the deliverables to make sure they fulfil the required quality. The same documents are also reviewed by the Quality Manager Assessor.\vspace{0.2cm} \\

	\midrule
	
	Quality Manager Assessor & Helps the Quality Manager in the tasks that he/she delegates.\vspace{0.3cm}\newline Reviews all the deliverables to make sure they fulfil the required quality. The same documents are also reviewed by the Quality Manager.\vspace{0.2cm} \\
	
	\midrule
	
	Technical Manager & Coordinates the work done by the engineers and technicians.\vspace{0.3cm}\newline Defines the technical quality plan and the software quality plan.\vspace{0.3cm}\newline Performs periodic quality controls on the design of the product.\vspace{0.3cm}\newline Reviews the technical aspects of the deliverables before approving them.\vspace{0.3cm}\newline Monitors the quality control procedures of both the prototype and the final product.\vspace{0.3cm}\newline Monitors the quality control procedures of the interaction platform.\vspace{0.2cm} \\
	
	\midrule
	
	Engineers and technicians & Make sure that the design of the product follows the technical quality plan.\vspace{0.3cm}\newline Perform quality control procedures over the prototype and over the final product.\vspace{0.3cm}\newline Make sure that the design of the interaction platform follows the software quality plan.\vspace{0.3cm}\newline Validate that the interaction platform fulfils the quality standards.\vspace{0.2cm} \\
	
	\bottomrule[2pt]
	
	\caption{List of quality roles and responsibilities}
	
\end{longtable}

\subsection{Risk Management Plan}

The present section will identify the risks threatening a project's correct development, the Work Packages that will be involved in such risks and will suggest a mitigation solution for each and everyone of them. The described information is found in the following table. The previously defined Work Packages apply for this section too.

\begin{longtable}[H]{p{4cm} p{4.7cm} p{5cm}}
	\toprule[2pt]
	
	\textbf{Description of risk} & \textbf{Work package(s) involved} & \textbf{Proposed risk-mitigation measure} \\
	
	\midrule[1.5pt] 
	\endhead
	
	Deliverable delays & WPx & Dedicate more resources than expected. \vspace{0.2cm} \\
	
	\midrule

	Inaccurate cost forecast & WP1, WP2 & Consider new funding sources and revise the financial management plan. \vspace{0.2cm} \\
	
	\midrule
	
	Lack of communication & WPx & Periodical meeting and use of collaborative software.  \vspace{0.2cm} \\

	\midrule

 	Lack of technology improvement & WP4 & Guarantee the development with thorough search of state of the art technologies.  \vspace{0.2cm} \\
 	
 	\midrule
 	
 	Lack of access to project needed information & WPx & Previous accurate research is needed before the development of the project. \vspace{0.2cm} \\
 
 	\midrule

	Low team motivation & WPx & Personnel control and team building projects. \vspace{0.2cm} \\
 	
 	\midrule

	Unsuccessful quality control & WPx & Improve or increase quality controls. \vspace{0.2cm} \\
 	
 	\midrule

	Conflicts between members  & WPx & Personnel conflict resolution meetings. \vspace{0.2cm} \\
 	
 	\midrule

 	Infeasible design & WP4 & Periodical reviews with experts and managers. \vspace{0.2cm} \\
 	
 	\midrule

	Techonolgical components with security vulnerabilities & WP3, WP4, WP5 & Check for possible security problems during development through specialised companies. \vspace{0.2cm} \\
 	
 	\midrule

	Organization issues & WPx & Demand for external companies specialised advice in project management. \vspace{0.2cm} \\
 	
 	\midrule

	Stakeholder desertion & WPx (Except WP7) & Transfer responsibilities to other stakeholders when possible or contract new ones. \vspace{0.2cm} \\
 	
 	\midrule

	Competitors appearance & WP6, WP7 & Improvement of the quality-to-price ratio of the service. \vspace{0.2cm} \\
 	
 	\midrule

	Delay in external deliverables & WPx & Control the delivery schedules and change provider if necessary. \vspace{0.2cm} \\
 	
 	\midrule

	Economical market issues & WPx & Control cost evolution due to external changes throughout the project. \vspace{0.2cm} \\
 	
 	\midrule

	Components or raw material quality & WP4, WP5 & Have exhaustive and regular quality controls to avoid problems in components in the final test. \vspace{0.2cm} \\
	
	\bottomrule[2pt]
	
	\caption{Critical risks for implementation}
	\label{workpackages}
\end{longtable}

\subsection{Communication Management}

Communication management has been effectively synthesized in a table where each and every type of communication process inside DEOS-UD is concisely described. This section will set an expanded overview regarding communication during the project, and will be helpful for all departments seeking for standardized guidance when trying to communicate results, problems or suggestions to other departments or to the general public. 

\begin{landscape}
	
	
	\begin{longtable}{| >{\raggedright\arraybackslash}p{2.8cm}  | >{\raggedright\arraybackslash}p{2.8cm} | >{\raggedright\arraybackslash}p{2cm} | >{\raggedright\arraybackslash}p{2cm} | >{\raggedright\arraybackslash}p{2cm} | >{\raggedright\arraybackslash}p{2.4cm} | >{\raggedright\arraybackslash}p{2.4cm} | >{\raggedright\arraybackslash}p{2.4cm} |  }
		
		
		\toprule [2pt]
		
		\textbf{Communication Type} & \textbf{Objective of Communication} & \textbf{Medium}  &\textbf{Frequency} &\textbf{Audience}& \textbf{Owner}& \textbf{Deliverable} &\textbf{Format} \\  
		
		\midrule [1.5pt]
		\endhead
		
		Internal Business Status Meetings& Discuss assignments, activities and sharing information  & Face to Face   &  Weekly &Business Team     & Financial Manager  & Agenda, Meeting Minutes  &Soft copy archived on SharePoint site and project website\\  
		
		\hline
		
		Technical and Business Status Meetings and Reports&Discuss assignments, activities, sharing information and reporting the project status   &Face to Face    & Weekly  & Project Manager, Business Team, Technical Team, Project Secretary    &Project Manager   & Agenda, Meeting Minutes, Status Reports  &Soft copy archived on SharePoint site and project website\\  
		
		\hline
		
		Advisory Committee Meetings& Review progress, risks and issues  & Face to Face   &Monthly   &  Advisory Committee, Project Stakeholders, Project Manager, Project Secretary   &Project Manager   & Agenda, Meeting Minutes  &Soft copy archived on SharePoint site and project website\\  
		
		\hline
		
		Steering Committee Status Meetings& Enhance communication and coordination of the project  & Face to Face   & Monthly  & Steering Committee, Project Manager, Project Secretary    &  Project Manager & Agenda, Meeting Minutes  &Soft copy archived on SharePoint site and project website\\  
		
		\hline
		
		Status Meetings and Reports to Stakeholders&Report the status of the project including activities, progress, costs and issues   &Face to Face or Video Conference    &Monthly   & Stakeholders, Project Manager, Project Secretary    & Project Manager  &Agenda, Meeting Minutes, Status Reports   &Soft copy archived on SharePoint site and project website\\  
		
		\hline
		
		Project Status Reports&Provide Stakeholders information on the status and progress of the project   &Email    & Monthly  &  Project Stakeholders, Stakeholder and Procurement Manager, Project Manager   & Stakeholder and Procurement Manager  &  Project status, schedule, budget and cost tracking, status of issues and risks, health status, status of action items, future or planned activities &Soft copy archived on SharePoint site and project website\\    
		
		\bottomrule[2pt]
		
		
		\caption{Communication management plan matrix}
	\end{longtable}
	
	\vspace*{\fill}
	
	
\end{landscape}