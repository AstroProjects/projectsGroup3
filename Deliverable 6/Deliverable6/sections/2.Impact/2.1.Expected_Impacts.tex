\section{Expected Impacts}

The following table shows how DEOS-UD will contribute to each of the expected impacts mentioned in the work programme:

\begin{longtable}{>{\raggedright\arraybackslash}p{3cm} p{11cm}}
	
	\toprule[2pt]
	
	\textbf{Impact} &  \textbf{Description}\\
	
	\midrule [1.5pt]
	\endhead
	
	Improvement in state-of-the-art technologies & DEOS-UD project will introduce machine learning philosophy, artificial intelligence, big-data collection and treatment to Earth Observation systems in order to achieve a better resolution and reactivity of observations. This will foster new technological research that will mean an improvement of the current state-of-the-art of that systems related to key areas of Earth Observation such as: 
	\begin{itemize}  
		\item Innovative LiDAR and radar instruments (including cost-effective wide-swath altimetry and imaging systems).
		\item Super-spectral and hyperspectral payloads with wide spectral and/or coverage, limb sounders and gravimetry payloads.
		\item High quantum efficiency photo-detectors and high-precision optical beam scanning and pointing.
		\item Advanced infrared (IR) technologies (optical filters, detectors and electronics).
	\end{itemize}\\
	
	\midrule
	
	Enhancement of Earth Observation capabilities & The improvement in state-of-the-art technologies for remote sensing will extend Earth Observation systems capabilities. DEOS-UD will especially advance in the fields of miniaturisation, power reduction, precision and efficiency.\vspace{0.2cm}\\
	
	\midrule
	
	Synergy among Earth Observation Constellations & All progress or development of the DEOS-UD project will be compatible with any of the current Earth Observation constellations. Moreover, the improved systems will be able to adapt to new possible EO constellations by using capabilities of machine learning and big data.\vspace{0.2cm}\\
	
	\midrule
	
	Improvement of Europe’s industrial competitiveness in Earth Observation technologies & DEOS-UD actively participates in the growth of the European industrial sector of Earth Observation since the stakeholders working on the project are European.\vspace{0.2cm}\\
	
	\midrule
	
	Greater industrial relevance of research actions & In the DEOS-UD project research will be conducted in the fields of payload, modular systems and Urban Development Applications with Space Technologies, including Weather forecast, urban planning (3D models) and greenhouse emissions reduction (pollution). The involvement of some of the stakeholders in this research will be necessary. Hence, results will be considered to be of greater relevance.\vspace{0.2cm}\\
	
	\midrule
	
	Better interrelation between academia and industry & Project research will not only have an influence from the industry but will also have the support of university researchers, always maintaining a rapid transfer of information and strengthening links between business and academia.\vspace{0.2cm}\\
	
	\bottomrule[2pt]
	
	\caption{Expected impact mentioned in the work programme}
\end{longtable}

\pagebreak

\subsection{Impact on market}

As said in the previous chapter, the result of this project is intended to play an important role in the performance of potential users belonging to urban development, such as the construction industry and governmental institutions. Its impact on the market in various fields is described below:

\begin{table}[H]
	\centering
	\begin{tabular}{l p{11cm}}
		
		\toprule[2pt]
		
		\textbf{Impact} &  \textbf{Description}\\
		
		\midrule [1.5pt]
		
		Innovative service & The call H2020 aims for a product that is innovative and contains an added value in environmental terms. This project will allow cities to grow in a sustainable way.\vspace{0.2cm}\\
		
		\midrule
		
		European industry & This proposal is a novelty in weather forecast, urban planning (3D models) and greenhouse emissions reduction (pollution), and will help cities to develop correctly, this being a concern that has increased in the recent years. This project is expected to lead this field.\vspace{0.2cm}\\
		
		\midrule
		
		City development & The incorporation of the platform will allow cities not to be affected by future environmental restrictions in the field of emissions. In this way, by using the platform, cities will be able to avoid future taxes or limitations that prevent them from developing properly.\vspace{0.2cm}\\
		
		\midrule
		
		Social & Because the construction industry has a relevant weight in the world and cities are constantly growing, it is expected that this platform will be extended to a large number of countries, creating new jobs.\vspace{0.2cm}\\
		
		\midrule
		
		Efficiency & The possibility of being able to track the emissions of cities and to build new scenarios will allow cities to grow in a more efficient way, optimizing their expansion within the allowed limits.\vspace{0.2cm}\\
		
		\midrule
		
		Environment & The use of this application will allow a precise tracking of emissions in cities, so that as cities continue to grow, they do it in a way that does not harm the environment, reducing the effects of climate change.\vspace{0.2cm}\\
		
		\bottomrule[2pt]
		
	\end{tabular}
	\caption{Expected impact on market}
\end{table}

\pagebreak

\subsection{Barriers and frameworks conditions}

Like any other project, DEOS-UD has to face a series of barriers or framework conditions that may affect its development or even prevent the project from being carried out. The following are the main barriers that can be found:

\begin{table}[H]
	\centering
	\begin{tabular}{l p{12cm}}
		
		\toprule[2pt]
		
		\textbf{Barrier} &  \textbf{Description}\\
		
		\midrule [1.5pt]
		
		Commercial & Since DEOS-UD is an innovative project that has just begun, its relevance will be weak at first, because the construction market is conservative and may take time to understand and accept the product. In this aspect a good commercial management is important, to guarantee that the potential clients see in the product a way to cover a need.
		\vspace{0.2cm}\\
		
		\midrule
		
		Technology & The project depends on the current state of various technologies, including satellite technology. It can happen that the required technological level of the project exceeds the one that currently exists, creating a problem in this aspect. It is important that the engineering department knows how to compensate these limitations and to comply with the requirements with the available technology.\vspace{0.2cm}\\
		
		\midrule
		
		Budget & The project corresponds to the call H2020, in which there is a financial threshold for its realization. It is important to keep in mind that, because the budget is higher than the proposed for this call, it is necessary to pay attention not to exceed it, taking special care in the part destined for contingencies.\vspace{0.2cm}\\
		
		\midrule
		
		Regulation & The aim of the company's quality department is to study all the regulatory frameworks in which the project operates. It is important to understand that there must be a regulation that can be more or less flexible, so it is necessary to adapt the project to it as much as possible or try to evolve it in the desired direction. However, due to the nature of the project, it is not expected to have too many drawbacks in this field.\vspace{0.2cm}\\
		
		
		\bottomrule[2pt]
		
	\end{tabular}
	\caption{Expected barriers and frameworks conditions}
\end{table}