\section{Expected Impacts}

\subsection{Impact on market}

The result of this project is intended to play an important role in the performance of potential users belonging to the construction industry. Therefore, the project will be a new way to develop cities. Its impact in various fields is described below:

\begin{table}[H]
	\centering
	\begin{tabular}{l p{13.3cm}}
		
		\toprule[2pt]
		
		\textbf{Item} &  \textbf{Description}\\
		
		\midrule [1.5pt]
		
		Innovative & The call H2020 aims that the result of the product is innovative and that it contains an added value in environmental terms. This project will allow cities to grow in a sustainable way.\vspace{0.2cm}\\
		
		\midrule
		
		Industry & This proposal is a novelty in the construction industry and will help cities to develop correctly, this being a concern that has increased in recent years, this project is expected to lead this field.\vspace{0.2cm}\\
		
		\midrule
		
		Growth & The incorporation of the platform to the construction industry will allow cities not to be affected by future environmental restrictions in the field of emissions. In this way, by using the platform, cities will be able to avoid future taxes or limitations that prevent them from developing properly.\vspace{0.2cm}\\
		
		\midrule
		
		Market & Because the construction industry has a relevant weight in the world and cities are constantly growing, it is expected that this platform will be extended by a large number of countries, created new jobs to be able to use the application.\vspace{0.2cm}\\
		
		\midrule
		
		Efficiency & The possibility of being able to control the emissions of the cities and to build new scenarios, will allow the cities to grow in a more efficient way, optimizing their expansion within the allowed limits.\vspace{0.2cm}\\
		
		\midrule
		
		Environment & The use of this application will allow a precise control of emissions in cities, this fact will make as cities continue to grow as they do so far in a way that does not harm the environment while avoiding climate change.\vspace{0.2cm}\\
		
		
		\bottomrule[2pt]
		
	\end{tabular}
	\caption{Technical requirements}
\end{table}

\subsection{Barriers and frameworks conditions}

Like any other project, it must face a series of barriers or framework conditions that may affect its development or even prevent the project from being carried out. The following are the main barriers that can be found:

\begin{table}[H]
	\centering
	\begin{tabular}{l p{13.3cm}}
		
		\toprule[2pt]
		
		\textbf{Item} &  \textbf{Description}\\
		
		\midrule [1.5pt]
		
		Commercial & Because it is an innovative project that has just begun, its relevance will be weak at first, because the construction market is conservative and may take time to understand and accept the product.
		In this aspect a good commercial management is important, it is necessary that the potential clients see in the product a way to cover a need.
		\vspace{0.2cm}\\
		
		\midrule
		
		Technology & The project depends on the current state of various technologies, including satellite technology. It can happen that the technological level of the project exceeds the one that currently exists, creating a problem in this aspect. It is important that the engineering department knows how to compensate these limitations and to comply with the requirements with the available technology.\vspace{0.2cm}\\
		
		\midrule
		
		Budget & The project corresponds to the call H2020, for which there is a financial threshold for its realization. It is important to keep in mind that, because the budget is a little exceeded. It is necessary to pay special attention not to exceed it more, taking special care in the part destined for contingencies.\vspace{0.2cm}\\
		
		\midrule
		
		Regulation & The aim of the company's quality department is to study all the regulatory frameworks in which the project operates. It is important to understand that there must be a regulation that can be more or less flexible, so you have to adapt the project to it as much as possible or try to evolve it in the desired direction. However, due to the nature of the project it is not expected that there are too many drawbacks in this field.\vspace{0.2cm}\\
		
		
		\bottomrule[2pt]
		
	\end{tabular}
	\caption{Technical requirements}
\end{table}
ning implementation, as covered in section 3.2.)