\section{Measures to maximise impact}

a) Dissemination and exploitation2 of results

Provide a draft ‘plan for the dissemination and exploitation of the project's results’. Please note that such a draft plan is an admissibility condition, unless the work programme topic explicitly states that such a plan is not required.

Show how the proposed measures will help to achieve the expected impact of the project.

The plan, should be proportionate to the scale of the project, and should contain measures to be implemented both during and after the end of the project. For innovation actions, in particular, please describe a credible path to deliver these innovations to the market.

Include a business plan where relevant.

As relevant, include information on how the participants will manage the research data generated and/or collected during the project, in particular addressing the following issues:
o What types of data will the project generate/collect?
o What standards will be used?
oHow will this data be exploited and/or shared/made accessible for verification and re-use? If data cannot be made available, explain why.
o How will this data be curated and preserved?
o How will the costs for data curation and preservation be covered?

Outline the strategy for knowledge management and protection. Include measures to provide open access (free on-line access, such as the ‘green’ or ‘gold’ model) to peer- reviewed scientific publications which might result from the project.

b) Communication activities

Describe the proposed communication measures for promoting the project and its findings during the period of the grant. Measures should be proportionate to the scale of the project, with clear objectives. They should be tailored to the needs of different target audiences, including groups beyond the project's own community.