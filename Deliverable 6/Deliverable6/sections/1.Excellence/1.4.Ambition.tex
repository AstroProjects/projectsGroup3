\section{Ambition}

Describe the advance your proposal would provide beyond the state-of-the-art, and the extent the proposed work is ambitious.

Describe the innovation potential (e.g. ground-breaking objectives, novel concepts and approaches, new products, services or business and organisational models) which the proposal represents. Where relevant, refer to products and services already available on the market. Please refer to the results of any patent search carried out.

---------------------------------------------------------------

As stated earlier, the main objective of the project is to enhance the performance of the EO systems so as to use the information derived from data to build a greener future. More specifically, the focus is on the improvement of both optical and radar systems and how can they contribute to the sustainable development of cities. 

To begin with, a research on the current technologies is carried out. This study makes it possible to determine which systems are more susceptible to further improvement. In order to demonstrate the advances in the aforementioned systems a prototype has to be manufactured and tested. 

Moreover, in the scope of this project, it has been included the development of a software that, once the data has been collected and received, treats the data in order to enable a more user-friendly data treatment on the final application and a web-based server for data sharing.

The project is grounded in initiatives such as the Copernicus programme. The Copernicus services aim at delivering nearly real-time data on a global level. This information allows us to better understand the planet we live in and secure a sustainable management of the environment. In fact, in context of the Copernicus, one of the previous H2020 calls has been involved in identifying possible potential evolutions of its space observation capabilities in order to build a climate resilient future. This call was focused on monitoring either the Polar Regions, agriculture or forests.

Among other things, Copernicus obtains data thanks to a set of dedicated satellites named Sentinel. Each of them has been developed for a specific need to provide accurate observation in each case. Nowadays, there is a total of six families of Sentinel. Hence, the idea is to take them a further step forward by equipping them with better remote sensing technologies. 