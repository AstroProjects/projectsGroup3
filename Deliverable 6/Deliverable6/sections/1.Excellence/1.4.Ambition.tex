\section{Ambition}

%\textit{Describe the advance your proposal would provide beyond the state-of-the-art, and the extent the proposed work is ambitious.
%Describe the innovation potential (e.g. ground-breaking objectives, novel concepts and approaches, new products, services or business and organisational models) which the proposal represents. Where relevant, refer to products and services already available on the market. Please refer to the results of any patent search carried out.}

%---------------------------------------------------------------

As stated earlier, an improvement of the state-of-the-art technologies used for EO sensing is a key factor to promote and advance in the Earth Observation field. In other words, this project is not in charge of developing new launching systems or designing satellites, its objective is to provide the existing and the next generation of space technologies with disruptive sensors. In fact, one of the priorities of this project is to ensure the complementarity with other activities or programs such as Copernicus funded.

Hence, the accomplishment of the project will demonstrate significant knowledge and enhancements concerning reliability, size, resolution, efficiency and accuracy among others of the current remote sensing technologies that will not only allow to gather better and more specific EO data, improving the results on their application fields but it will also represent a step forward in all those areas involving remote sensing from which the European society will benefit. 

The project is grounded in initiatives such as the Copernicus programme. The Copernicus services aim at delivering nearly real-time data on a global level. This information allows us to better understand the planet we live in and secure a sustainable management of the environment. In fact, in context of the Copernicus, one of the previous H2020 calls has been involved in identifying possible potential evolutions of its space observation capabilities in order to build a climate resilient future. This call was focused on monitoring either the Polar Regions, agriculture or forests.

Among other things, Copernicus obtains data thanks to a set of dedicated satellites named Sentinel. Each of them has been developed for a specific need to provide accurate observation in each case. Nowadays, there is a total of six families of Sentinel. Hence, the idea is to take them a further step forward by equipping them with better remote sensing technologies. 