\section{Relation to the work programme}
The current proposal relates to the topic "Earth Observation technologies" whose identifier is: LC-SPACE-14-TEC-2018-2019. More specifically it addresses the subtopic "Disruptive technologies for remote sensing". 

Hence, this project aims to research and improve the existing EO technologies for remote sensing, develop a data processing software along with it containing machine learning algorithms focused on urban sustainable developments such as pollution and gas emission control, traffic monitoring, weather prediction, management of urban areas, regional and local planning, tourism development and cityscapes designs, and develop a web based server for data sharing. 

To begin with, a research on the current technologies is carried out. This study makes it possible to determine which systems are more susceptible to further improvement. In order to demonstrate the advances in the aforementioned systems a prototype has to be manufactured and tested. Moreover, in the scope of this project, it has been included the development of a software that, once the data has been collected and received, processes the data in order to enable a more user-friendly data treatment on the final application and a web-based server for data sharing.

The implemented data processor will provide information sets about sustainable development issues such as geospatial indicators, pollution levels or gas emissions that will benefit companies and initiatives from world-wide and local organisations to carry out social and green actions, and will support the United Nation projects: UN 2030 Agenda for Sustainable Development and The Paris Agreement on Climate Change. Furthermore the project sharing web will allow the public to interact, enriching and contributing in the integration of space in economy and society.
 
Additionally, the attainment of the improved sensors and data processing software is expected to help process the data gathered by the Sentinel satellites in order to benefit the current on-going Copernicus programme missions so as to equip them with better remote sensing technologies in the near future.  