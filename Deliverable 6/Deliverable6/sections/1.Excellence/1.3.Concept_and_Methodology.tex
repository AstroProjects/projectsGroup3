\section{Concept and methodology}


%(a) Concept

%Describe and explain the overall concept underpinning the project. Describe the main ideas, models or assumptions involved. Identify any inter-disciplinary considerations and, where relevant, use of stakeholder knowledge. Where relevant, include measures taken for public/societal engagement on issues related to the project. Describe the positioning of the project e.g. where it is situated in the spectrum from ‘idea to application’, or from ‘lab to market’. Refer to Technology Readiness Levels where relevant.

%Describe any national or international research and innovation activities which will be linked with the project, especially where the outputs from these will feed into the project;

%(b) Methodology

%Describe and explain the overall methodology, distinguishing, as appropriate, activities indicated in the relevant section of the work programme, e.g. for research, demonstration, piloting, first market replication, etc.

%Where relevant, describe how the gender dimension, i.e. sex and/or gender analysis is taken into account in the project’s content.

%-----------------------------------------


Earth observation is a field with a great potential that has not been taken into account until the last decade. Important space agencies like the European Space Agency are promoting the enhancement of capabilities with respect to Earth Observation due to the fact the society and the planet itself would benefit from the many application it has. Besides, Earth Observation can have much application, so it is crucial to focus on the enrichment of some of them to guarantee the development of the desired sensor abilities. Indeed, as the goal is to apply EO sensing for Urban Development to integrate space into
society, the abilities to enhance are the following ones:

\begin{itemize}
 \item  Detection of greenhouse gases.
 \item  Detection of weather patterns.
 \item  High precision performance of terrain 3D mapping.
\end{itemize}

On the one hand, systems like LiDAR, which combines technologies like laser and radar, enable to target a wide range of materials including clouds and molecules. Consequently, it is possible to develop a sensor that identifies the composition of the air to secure our environment by having a monitoring of either the greenhouse gases or the weather patterns for proper weather forecasting applications. On the other hand, 3D mapping of the terrain is useful to control the land and guarantee an optimum growth and development of the city. All in all, one of the most important aspects that have to be taken into account is that the sensors resulting from this project have to ensure at least a 15\% increase of the reliability and precision compared to the current ones.

To achieve the project goal and implement much better sensors than the already existing ones, a state-of-the-art of the current space requirements of several optical and radar systems will be done. The limitations and the possibilities of the different technologies such as LiDAR, RADAR, Gravimetry, Hyperspectral, Superspectral and more will be determined, and then a decision will be taken in order to work with the most promising ones. Furthermore, the preliminary design will take into account several criteria to obtain competitive sensors. Launching any payload to space has very high costs, then it is essential to ensure the endurance of the overall systems in order to maintain the payload in space for a long time and avoid any replacements. To accomplish it, the materials used to build the components of the sensor including antennas, photo-detector, optics, laser and, electronics have to be accurately chosen.

In addition, a step that is necessary for this kind of projects is the testing of the product. Once the preliminary design is finished an accomplishes all the requirements of the project, a first prototype will be built and tested in a space simulated environment to make sure that it performs as expected. Notice that the testing is not done in the space itself because launching the prototype to the space is too expensive and out of this project budget; fortunately, there are other methods that are cheaper and simulate properly the space conditions. Finally, once the prototype designed fulfils all the expectations, it is considered that the results are attained and the product design is ready for closure.