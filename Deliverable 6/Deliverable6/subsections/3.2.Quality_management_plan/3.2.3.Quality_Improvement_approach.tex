\subsubsection{Quality Improvement Approach}
Quality improvement (QI) is a formal analysis of practice performance and efforts done in order to improve the performance of the project with the main objective of increasing its efficiency. The information shown about QI models and tools has been extracted from \cite{aafp} and \cite{leansolutions}. A proper QI process requires some basics to succeed. These basics are the following ones:
\begin{itemize}
\item Establish a culture of quality in the project: Creation of QI teams, QI meetings, and QI goals.
\item Determine and prioritize potential areas of improvement: Define, according to the acceptance criteria of the project, the main areas of improvement.
\item Collect and analyse data: Determine the type of data to be collected and analyse it properly according to the project objectives.
\item Communication of results: Quality improvements should be transparent to the stakeholders in order to keep them satisfied. 
\end{itemize}
In this project, the six-sigma working philosophy will be implemented in order to improve quality. The objective of this philosophy is to adjust the existing processes in order to improve the quality and minimize the variability by reducing defects and irregularities. The model related to six-sigma philosophy that will be used is DMAIC. This model includes the following steps:
\begin{itemize}
\item Define: Set the objective of the problem or the existent defect. In this project, this definition will be done according to the acceptance criteria. The improvement of the quality plan is one of the objectives that will need to be taken into account.
\item Measurement: Measures are needed in order to have values for the problem or defect. In this project the measurements according to the effectiveness of the quality plan are:
\begin{itemize}
\item Number of iterations of a document/design to be approved.
\item Stakeholders satisfaction
\item Time needed to approve a document/design.
\item Number of defects detected by the quality department 
\end{itemize}
\item Analyse: Figure out the causes of the problem or defect and propose solutions.
\item Improve: Implement the approved solution.
\item Control: Control the implementation of the improvement, assure continuity and success.
\end{itemize}